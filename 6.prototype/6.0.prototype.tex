Für den praktischen Anteil dieser Arbeit wird auf Grundlage der gewonnenen Erkenntnisse aus vorheriger Sektionen ein Prototyp angefertigt. Es wird also auf die Hypothesen zurückgegriffen, welche durch die Interviews und Umfragen erarbeitet und überprüft wurden. Außerdem wird anhand eines geeigneten Beispiels der Natur eine Thematik Stück für Stück untersucht und in Mechaniken transformiert. Konkret werden in Folgendem die Vorgänge innerhalb einer Bienenkolonie genauer beleuchtet, und erörtert, welche Elemente sich für ein Resource Management Game eignen. Der Prototyp wird versioniert mittels GitHub und ist jederzeit in Form eines \href{https://github.com/ProgFroz/hexabees}{Repositories} \cite*[]{repository} aufrufbar.

\subsection{Thematik}
Für den Prototypen bedarf es einer Thematik beziehungsweise einer Idee für das Gesamtkonzept. Dafür wurde die Idee einer \textit{Bienenkolonie} für interessant befunden. Bienen und deren Kolonien sind aus vielerlei Hinsicht geeignet als Thematik für den Prototyp, darunter der Fakt, dass Bienen bekannt dafür sind, Honig zu produzieren, wobei Honig eine konkrete Ressource darstellt. Im Kontext eines Resource Management Games und mit Rücksicht auf das zuvor untersuchte Spiel \textit{RimWorld} eignet sich jedoch am besten eine ganz spezifische Art der Bienen, die \textit{Honigbiene}, genauer gesagt die \textit{Westliche Honigbiene} (lat. \textit{apis mellifera}) \cite*[]{bees:name}. Grund dafür ist die Verhaltensweise von Honigbienen, da diese, anders als nahe Verwandte, stärker dazu tendieren, in Kolonien zu leben. Hummeln beispielsweise formen zwar auch Kolonien, aber deutlich kleinere und kürzer lebende. Die meisten Arten der Wildbienen jedoch sind tendenziell Einzelgänger. Eine weibliche Wildbiene legt ein Nest und versorgt ihren Nachwuchs mit Pollen und Nektar \cite*[]{bees:wild}. Somit bietet die Gattung der Honigbiene einige Vorgänge, welche sich in ein Colony Management Game umsetzen lassen. Dazu werden im Folgenden interessante Vorgänge und Eigenschaften der Gattung der Honigbiene genauer untersucht. Es ist zu erwähnen, dass der folgende Abschnitt nicht dazu dient, das gesamte Spektrum einer Bienenkolonie zu untersuchen, sondern lediglich Aspekte, die förderlich für die Entwicklung des Resource Management Games wären.

\subsubsection{Nahrung}
Eine Honigbiene ernährt sich primär von \textit{Nektar}, welchen sie von Blumen sammelt. Dieser Nektar ist zuckerhaltiger Saft, welcher von den Pflanzen ausgeschieden wird, um damit Insekten verschiedener Arten anzulocken. Dabei sind Sonnenblumen, Obstblüten, Löwenzahn und Raps besonders gute Nektarquellen \cite*[]{bees:food}. Eine weitere Möglichkeit um an energiereichen Zucker zu gelangen ist der sogenannte \textit{Honigtau}, welcher von einigen Insekten, darunter Blatt- und Schildläusen, ausgeschieden wird. Aus diesen beiden zuckerhaltigen Säften können die Bienen den \textit{Honig} produzieren, welcher ebenfalls als Nahrungsquelle dient, aber vor allem über die Wintermonate besonders wichtig ist, da Honig sehr lange haltbar ist \cite*[]{bees:honeywinter}. Der durch Honigtau produzierte Honig wird von Imkern als \textit{Waldhonig} gekennzeichnet \cite*[]{bees:food}. Saugt eine Biene den Nektar einer Blüte auf, gelangt dieser in den \textit{Honigmagen}. Dort kommt er in Kontakt mit verschiedenen Enzymen, welche den Nektar in \textit{Glukose} (Traubenzucker) und \textit{Fruktose} (Fruchtzucker) \cite*[]{bees:glucosefructose} aufspalten. Diesen biochemisch veränderten Nektar gibt die \textit{Sammelbiene} im Bienenstock an die sich dort befindlichen \textit{Arbeitsbienen} durch Auswürgen weiter, welche den Nektar erneut aufnehmen und auswürgen. Mit jedem dieser Prozesse wird der Nektar viskoser, wodurch allmählich Honig entsteht \cite*[]{bees:nectartohoney}.

Neben dem Nektar, dem Honigtau und dem Honig, welche primär als Quellen für Kohlenhydrate dienen, benötigen Bienen außerdem noch Quellen für Fette und Proteine. Dafür werden \textit{Pollen} verwendet, wobei Pollen die männlichen Geschlechtszellen einer Samenpflanze sind. Da die Samenpflanze mehr Pollen produziert, als für die Fortpflanzung nötig, können diese ohne Probleme von den Sammelbienen aufgenommen werden \cite*[]{bees:honeywinter}. Die Pollen werden in \textit{Pollenpaketen} aufbewahrt (vgl. \autoref{image:pollenpaket}), wobei Teile der aufgesammelten Pollen nicht in diesen Pollenpaketen landen, sondern an den Hinterbeinen haften bleiben. Besucht diese Sammelbiene nun eine weitere Blüte, werden diese Pollen abgestreift und der Fortpflanzungsmechanismus der Samenpflanze profitiert. Besonders pollenreiche Quellen sind dabei Obstbäume, Mohn, Mais, Klee und Raps \cite*[]{bees:honeywinter}.

\begin{figure}
    \begin{center}
        \includegraphics[width=300px]{0.bilder/pollenpaket.jpg}
    \end{center}
    \caption{Pollenpakete einer Honigbiene (\cite{bees:name})} \label{image:pollenpaket}
\end{figure}

Die letzte wichtige Ressource, welche Bienen für ihr Überleben benötigen, ist Wasser. Dieses kann von Flüssen oder Seen bezogen werden, wobei diese Quellen nicht mehr als 500 Meter weit vom Stock entfernt sein sollten. Erhalten Bienen nicht genug Wasser, können diese unter Verstopfungen leiden, was für das Überleben gefährlich sein kann \cite*[]{bees:honeywinter, bees:food}.

Es lassen sich somit wichtige Ressourcen identifizieren: \textit{Nektar, Honigtau, Honig, Pollen} und \textit{Wasser}. Außerdem ist die Aufteilung zwischen \textit{Sammel-} und \textit{Arbeitsbiene} eine mögliche Mechanik. Die Umwandlung von Nektar und Honigtau zu Honig könnte ebenfalls eine Mechanik darstellen, wie auch die naturgegebenen Quellen für Nektar, Honigtau und Pollen.

\subsubsection{Kasten}
Eine fundamentale Kategorisierung innerhalb einer Kolonie sind die \textit{Kasten}. Es gibt drei verschiedene Kasten beziehungsweise Arten von Bienen innerhalb solch einer Kolonie. Die erste Art ist die \textit{Bienenkönigin}. Diese Art der Biene ist genau ein mal vertreten und \textit{immer} ein Weibchen, zudem innerhalb einer Kolonie das einzige vollentwickelte. Die Hauptaufgabe der Königin ist das Brüten neuer Bienen und das Steuern des Schwarms mittels verschiedener Pheromone \cite*[]{bees:queen}. Die zweite Art sind die sogenannten \textit{Drohnen}. Diese Art ist \textit{ausschließlich männlich}. Diese Kaste ist lediglich zur Befruchtung der Königin da und ist nur von Frühling bis Sommer des Jahres in der Kolonie vertreten. Anschließend werden alle Drohnen gewaltsam aus der Kolonie entfernt, was als \textit{Drohnenschlacht} betitelt wird \cite*[]{bees:sex}. Die dritte und letzte Kaste der Kolonie sind die \textit{Arbeitsbienen}, welche, analog zur Königin, \textit{ausschließlich weiblich} sind, jedoch den deutlich größten Teil einer Kolonie ausmachen. Trotzdem legen diese Bienen in der Regel keine Eier, sind im Gegensatz dazu aber deutlich fürsorglicher gegenüber der Brut im Vergleich zur Königin. Die Arbeitsbienen sind prinzipiell für sämtliches weiteres Geschehen verantwortlich, darunter die Entfernung von Leichen oder die Nahrungsbeschaffung \cite*[S.2-3]{bees:frisch}. In \autoref{image:castes} werden die verschiedenen Kasten morphologisch unterscheidbar dargestellt mit einer Markierung verschiedener Merkmale. 

\begin{figure}
    \begin{center}
        \includegraphics[width=300px]{0.bilder/castes.png}
    \end{center}
    \caption{(a) Königin, (b) Arbeitsbiene, (c) Drohne. (K) Kopf, (B) Brust, (H) Hinterleib, (A) Auge, (F) Fühler (\cite[S.2]{bees:frisch})} \label{image:castes}
\end{figure}



\subsubsection{Fortpflanzung}
Ein wichtiger Teil des Fortbestehens einer Kolonie ist die \textit{Vermehrung} beziehungsweise \textit{Fortpflanzung}. Wie bereits zuvor erwähnt gilt, dass sowohl Königinnen, als auch Arbeitsbienen jederzeit \textit{weiblich}, und Drohnen stets \textit{männlich} sind. Eine Königin ist in der Lage unbefruchtete Eier zu legen, oder diese Eier mittels Paarung mit Drohnen zu befruchten. Auch Arbeitsbienen sind in der Lage unbefruchtete Eier zu legen, was in einer Kolonie tendenziell selten passiert, da die Geschlechtsorgane der Arbeitsbienen zurückentwickelt und verkrümmt sind \cite*[]{bees:sex}. Wird allmählich eine neue Königin gebraucht, da die derzeitige Königin an das Ende ihres Lebens gelangt, sondert diese bestimmte Pheromone aus, wodurch dem Schwarm mitgeteilt wird, neue Königinnen heranzuziehen. Eine würde theoretisch ausreichen, aber um kein Risiko einzugehen werden mindestens \textit{sechs} Königinnen, oder mehr, versucht heranzuziehen. Nachdem eine geeignete Königin geschlüpft ist, werden die restlichen gewaltsam aus dem Schwarm entfernt \cite*[S.33]{bees:frisch}. Spätestens zwei Wochen später fliegt die neu geschlüpfte Königin aus und versprüht Pheromone, auch \textit{Königinnensubstanz} genannt, welche Drohnen eigener und fremder Völker anlocken, um sich mit der Königin zu paaren. Es wird mit maximal 12 Drohnen der Paarungsakt vollzogen, wobei die Königin bis zu \textit{zehn Millionen} Spermien aufnimmt, welche für die restliche Lebenszeit reichen \cite*[]{bees:queen}.

Einige Tage nachdem die Eier der zukünftigen Königinnen gelegt wurden, in der Regel neun Tage, beginnt der Akt des \textit{Schwärmens}. Dabei verlässt ein Teil der Kolonie, tausende von Bienen, zusammen mit der noch regierenden Königin den Schwarm, um sich auf die Suche nach einem neuen Zuhause zu machen und ein neues Bienenvolk zu gründen. Dieser Akt passiert in der Regel nur ein Mal pro Jahr, gegen Mai. \textit{Spurbienen} agieren als Späher und teilen Informationen über mögliche neue Heimatplätze. Laut Forschungen bedarf es 15 Spurbienen, welche dieselben Informationen über einen passenden Ort teilen, damit die Entscheidung gefällt wird \cite*[]{bees:swarm}. Dank der Pheromone der Königin bleibt der Schwarm während des Schwärmens stets zusammen \cite*[]{bees:queen}. 

Eine Königin besitzt einen \textit{diploiden} Chromosomensatz, dementsprechend \textit{2n = 32} Chromosomen. Die \textit{unbefruchteten} Eier werden von keinem zweiten Chromosomensatz ergänzt, wodurch die sich daraus entwickelnden Drohnen vorerst einen \textit{haploiden} Chromosomensatz besitzen, dementsprechend \textit{n = 16} Chromosomen in Summe. Diese werden jedoch nachträglich diploid, lediglich die Keimzellen bleiben haploid. Dadurch gilt, dass die anschließend diploiden Körperzellen der Drohne  \textit{homozygot} sind, da sie mittels \textit{Autopolyploidisierung} aus einer haploiden Eizelle entstanden sind. Beide Gene eines Merkmals stimmen also exakt überein \cite*[]{bees:homozygot}. Aus unbefruchteten Eiern schlüpfen ausschließlich Drohnen. Gegensätzlich dazu schlüpfen aus den \textit{befruchteten} Eiern Arbeitsbienen \textit{oder} Königinnen. Diese Eizellen sind durch die Beteiligung zweier Partien, der Königin und einer Drohne, sowohl \textit{heterozygot}, als auch diploid. Der Faktor, ob aus einem befruchteten Ei eine Arbeitsbiene oder eine Königin schlüpfen wird, ist modifikativ, also durch äußere Einflüsse, bestimmt. Der Unterschied liegt in der zugegebenen Nahrung der Maden in den ersten drei Lebenstagen. Während die zukünftigen Arbeitsbienen mit Pollen und Honig ernährt werden, erhalten die zukünftigen Königinnen ausschließlich \textit{Gelée Royal} (engl. \textit{Royal Jelly}) in ihren sogenannten \textit{Weiselzellen}, welche ausschließlich einer zukünftigen Königin vorenthalten sind \cite*[]{bees:swarm}. Somit gilt, dass Arbeitsbienen und Königinnen genetisch identisch sind, jedoch durch die zugegebene Nahrung modifikativ zu einer gewissen Kaste herangezogen werden können \cite*[]{bees:sex}. Es gilt also dass, auch wenn auf den ersten Blick paradox erscheinend, eine Drohne nie einen Vater, aber immer einen Großvater hat.

\subsubsection{Metamorphose}
Der Vorgang der \textit{Metamorphose} beschreibt das Anpassen der physischen Form oder ein plötzliches Ändern der äußeren Erscheinung \cite*[]{bees:metamorphosisdefinition}. Bienen durchlaufen vier verschiedene Zustände der Metamorphose, beginnend als \textit{Ei} (engl. \textit{egg}). Diese Eier werden von der Königin in eine leere Wabe gelegt, haben eine Größe von 1mm bis 1.5mm und ähneln einem Reiskorn. Nach circa \textit{drei Tagen} brechen die Eier und eine \textit{Larve} (engl. \textit{larva}) kommt hervor. Diese Larve ist weiß und C-förmig (vgl. \autoref{image:metamorphosis}). Die Dauer der kommenden Metamorphose hängt von der jeweiligen Kaste der zukünftigen Biene ab, Arbeitsbienen benötigen 6 Tage, Drohnen 6.5 Tage und Königinnen 5.5 Tage. Ist die Larve reif für die Metamorphose, richtet sie sich auf, sodass die Arbeitsbienen, welche Zuständig für die Brut sind, die Zelle mit \textit{Bienenwachs} bedecken. Die Larve verhärtet und wird zu einer \textit{Puppe} (engl. pupa). Analog zum vorherigen Stadium ist die Dauer bis zur kommenden Metamorphose abhängig von der Kaste der zukünftigen Biene. Eine Arbeitsbiene benötigt 12 Tage, eine Drohne 14.5 Tage und eine Königin 8 Tage. Ist die Zeit reif und die Metamorphose abgeschlossen beißt sich die Biene durch das Bienenwachs der Zelle und gesellt sich zu ihren Artgenossen \cite*[]{bees:name}.


\begin{figure}
    \begin{center}
        \includegraphics[width=300px]{0.bilder/metamorphosis.jpg}
    \end{center}
    \caption{Verschiedene Stadien einer Honigbiene, ausgewachsen (links), Puppe (mitte), Larve (rechts) (\cite[]{bees:name})} \label{image:metamorphosis}
\end{figure}

\subsubsection{Aufgabenverteilung}
Die Aufgaben der Königin und der Drohnen wurde zuvor bereits ausgiebig erläutert.
Eine Arbeitsbiene hat drei exakt eingeteilte Phasen ihres Lebens, in welchen verschiedene Tätigkeiten verrichtet werden. Die zu erledigenden Aufgaben sind also direkt gekoppelt an das derzeitige Alter einer Arbeitsbiene. Der erste Lebensabschnitt ist zwischen dem 1. und 10. Lebenstag der Biene. In diesem Stadium wird die Biene als \textit{Hausbiene} bezeichnet, da sie sich 
ausschließlich innerhalb des Bienenstocks aufhalten. Dort reinigen sie die Zellen, kümmern sich um die Brut als sogenannte \textit{Brutamme} und sind ansonsten meist untätig. Der zweite Lebensabschnitt findet zwischen dem 10. und 20. Lebenstag der Biene statt. In diesem Abschnitt wird die Biene als \textit{Baubiene} eingestuft. Die Aufgaben sind nun primär das Entgegennehmen des von den Sammlern gebrachten Nektar oder der Pollen, das Bauen neuer Waben und das Herstellen von Wachs. Außerdem sind manche diese Bienen im \textit{Wächterdienst} und beschützen den Stock vor Eindringlingen wie Wespen, Pferden oder Menschen. Im dritten und letzten Abschnitt des Lebens sind die Bienen \textit{Sammlerbienen}, welche Nahrung für den restlichen Stock von der Außenwelt suchen. Allerdings gibt es nur Ausflüge, wenn das Wetter und die Temperatur passend sind. Bei Regen, Schnee oder generell im Winter sitzen diese Bienen im Stock und warten auf Veränderung der Außenbedingungen. In der Regel widmen diese Bienen sich nicht den anderen Aufgaben und warten stattdessen \cite*[S.42-44]{bees:frisch}. \autoref{image:livesofcastes} zeigt einen Überblick der jeweiligen Aufgaben abhängig von dem Alter einer Biene jeglicher Kaste.

\begin{figure}
    \begin{center}
        \includegraphics[width=400px]{0.bilder/livesofcastes.png}
    \end{center}
    \caption{Überblick über den Lebensverlauf einer Biene jeder Kaste (\cite[]{bees:lifeexpectancy})} \label{image:livesofcastes}
\end{figure}

\newpage
\subsubsection{Lebenserwartungen}
Die Lebenserwartung einer jeden Biene hängt mit ihrer jeweiligen Kaste zusammen. Die kürzeste Lebensdauer weisen die Drohnen auf, welche zwischen zwei bis vier Wochen leben, da diese anschließend durch die Drohnenschlacht gewaltsam entfernt oder getötet werden. Am längsten hingegen leben Königinnen, mit drei bis fünf Jahren Lebensdauer. Bei Arbeitsbienen ist es entscheidend, ob diese während des Sommers oder gegen Anbruch des Winters geschlüpft sind. Durch die Wintereinnistung steigt die Lebenserwartung der Arbeitsbienen stark an, von gerade mal zwei bis vier Wochen auf sechs bis sieben Monate \cite*[]{bees:lifeexpectancy}. Eine Übersicht dieser Lebenserwartungen ist in \autoref{image:lifeexpectancy} vorzufinden.

\begin{figure}
    \begin{center}
        \includegraphics[width=400px]{0.bilder/lifeexpectancy.png}
    \end{center}
    \caption{Die minimale und maximale Lebenserwartung einer Biene jeder Kaste (\cite[]{bees:lifeexpectancy})} \label{image:lifeexpectancy}
\end{figure}

\subsubsection{Winter}
Die Vorbereitung auf den kommenden Winter beginnt bereits im Spätsommer, wobei neue Brut erzeugt wird für den Zweck der Überwinterung, sogenannte \textit{Winterbienen}. Ab Oktober werden die \textit{Sommerbienen} aus der Kolonie entfernt, sodass die Kolonie lediglich aus Winterbienen besteht. Dadurch, dass keine Energie für Brutpflege oder Sammelflüge aufgewendet werden muss, ist die Lebenserwartung der Winterbienen deutlich höher als die der Sommerbienen, wie bereits zuvor erläutert. Außerdem legen sich die Winterbienen ein Fettpolster zu, durch hohen Konsum von Pollen. Statt eines Winterschlafes formiert sich der Stock zu einer \textit{Wintertraube}, bei welcher die Kolonie die Königin umschließt und durch Muskelvibration warm hält. Mit Anfang des Frühlings wird mit gleicher Technik der Stock auf 35°C erhitzt, wodurch der Nahrungsverbrauch stark ansteigt aber das erfolgreiche Brüten der neuen Generation sichert. Sobald die ersten Blumen wieder sprießen setzen wieder die Sammelflüge ein \cite*[]{bees:winter}.
\subsection{Spiel-Engine}
Ein Videospiel wird heutzutage nicht mehr von Grund auf neu programmiert. Die Grundlage schafft in den meisten Fällen eine sogenannte \textit{Game Engine} (Spiele-Engine). Diese Engine bietet wichtige Grundlagen, welche es dem Entwickler oder dem Entwicklerteam deutlich vereinfachen loszulegen. Darunter fallen Grafik-, Physik- und Audiosysteme, welche je nach Engine variieren \cite*[]{whygameengine}. Aufgrund der sehr hohen Popularität \cite*[]{mostusedengines} und der bereits bestehenden Erfahrung wird bei dieser Arbeit auf \textit{Unity}\cite*[]{unity} zurückgegriffen. Mögliche und erwogene Alternativen waren \textit{Godot Engine} und \textit{Unreal Engine}, welche sich rein theoretisch allesamt anbieten würden. Auch wenn jede dieser Game Engines für den Umfang und die Komplexität des Prototypen geeignet wäre, überwiegt die vorhandene Erfahrung in Unity für die letztendliche Entscheidung.
\subsection{Spielwelt}
Die grundlegende Idee der Spielwelt ist, dass diese in zwei Instanzen geteilt wird. Die erste Instanz stellt die Oberwelt dar, wo Pflanzen und Bäume wachsen, wobei die zweite Instanz den Innenraum des Bienenstocks darstellt. In dem Bienenstock können Strukturen gebaut werden, welche in der Außenwelt nicht verfügbar sind. In der Außenwelt hingegen finden sich die Ressourcen, welche für das Fortleben essenziell sind. Für die gesamte Welt, in welcher das Spiel gespielt wird, stehen mehrere Optionen der Erstellung zur Verfügung.

\paragraph{Gridless}
Die erste Möglichkeit einer Spielwelt ist eine Karte ohne dargestelltes oder vorhandenes Grid. Solch ein System verwendet beispielsweise \textit{Age of Mythology}, wobei die Karte nicht in ein Grid, sondern viele tausende  Koordinaten eingeteilt ist. Diese Form der Positionierung ist daher gängig für \textit{Realtime Strategy} Spiele.

\paragraph{Square Grid}
Eine weitere Möglichkeit zur Positionierung von Gebäuden oder Einheiten innerhalb einer Spielwelt ist ein \textit{Square Grid} (vgl. \autoref{image:squaregrid}), also die Einteilung der Karte in eine gewisse Anzahl von Quadraten, welche als eigene Felder agieren und worauf Gebäude oder Einheiten zugreifen können. Dieses System wird unter anderem in \textit{Starcraft II} oder \textit{Age of Empires} verwendet, wodurch auch diese Form der Positionierung gängig für \textit{Realtime Strategy} ist. Allerdings findet man diese Eigenschaft auch in \textit{Turn Based Strategy}, darunter die älteren Teile der \textit{Civilization}-Reihe. Dieses System wurde im Laufe der Jahre jedoch von einem Square Grid auf ein Hex Grid umgestellt.

\begin{figure}
    \begin{center}
        \includegraphics[width=300px]{0.bilder/squaregrid.png}
    \end{center}
    \caption{Aufbau eines Square Grids (\cite{world:grids})} \label{image:squaregrid}
\end{figure}

\paragraph{Hex Grid}
Die letzte Möglichkeit ist ein \textit{Hex Grid}, welches vor allem Anklang findet im Genre der \textit{Turn Based Strategy} Games, darunter \textit{Endless Legend}, \textit{Civilization VI} und \textit{Humankind}. Der klare Vorteil von einem Grid bestehend aus Hexagonen ist die \textit{Distanzberechnung} und die Anzahl \textit{direkter Nachbarn}. Im Vergleich zu einem Quadrat besitzt ein Hexagon in einem Grid 6 statt 4 direkter Nachbarn, wobei direkt bedeutet, dass die Kanten aneinander angrenzen. Um einen diagonalen Weg in einem Square Grid einzuschlagen, muss ein weiterer Weg aufgewendet werden, da nach dem \textit{Satz des Pythagoras} gilt:
\begin{equation}
    a^2 + b^2 = c^2
\end{equation}
Der diagonale Weg innerhalb eines Quadrates mit Länge 1 und Breite 1 wäre folglich 
\begin{equation}
    \sqrt{1^2 + 1^2} = \sqrt{2}
\end{equation}
Wobei offensichtlich gilt, dass
\begin{equation}
    \sqrt{2} > 1
\end{equation}
Möchte man also Bewegungen von Einheiten in 6 statt 4 Richtungen ermöglichen, empfiehlt sich ein Hex Grid statt einem Square Grid. Die Distanz zwischen allen gegenüberliegenden Kanten eines Hexagons ist stets gleichlang (vgl. \autoref{image:hexgrid}). Unter der Berücksichtigung, dass die Thematik der Bienen auch mit \textit{Bienenwaben} und deren hexagonalen Form assoziiert wird, wird der Prototyp ein Hex Grid verwenden. 
\begin{figure}
    \begin{center}
        \includegraphics[width=300px]{0.bilder/hexgrid.png}
    \end{center}
    \caption{Aufbau eines Hex Grids (\cite{world:grids})} \label{image:hexgrid}
\end{figure}

\subsubsection{Generierung}
Das Terrain beziehungsweise Grid wird über einen vielschichtigen Algorithmus generiert, welcher die gesamte Fläche in Columns und die Columns wiederum in Chunks unterteilt. Der Algorithmus wurde mittels eines im Internet gefundenen Tutorials \cite*[]{world:tutorial} erarbeitet und anschließend angepasst. Mittels der Methode \textit{GenerateMap} (vgl. \autoref{code:worldgeneration}) kann von außerhalb dann der Algorithmus verwendet werden. Es gibt etliche Variationen zur Anpassung, darunter ein Seed, welcher die Generierung beeinflusst, Meeresspiegel, Erosion, Feuchtigkeit und Temperatur, es ist damit viel Diversität zwischen verschiedener Karten gegeben. In dieser Arbeit wird nicht weiter auf dieses Tutorial eingegangen, ist jedoch für weitere Informationen verlinkt. Der Bienenstock ist in dem Prototypen lediglich eine kleinere Version der generierten Außenwelt und noch nicht visuell als Bienenstock erkennbar. Dies wird in zukünftigen Iterationen angepasst und gelb und orange eingefärbt. Außerdem soll ein Ausgang für die Bienen in dem Bienenstock generiert werden, durch welchen die Einheiten sich von Bienenstock nach Außenwelt und umgekehrt bewegen können. 

\definecolor{LightGray}{gray}{0.9}
\begin{listing}[H]
\caption{World Generation}
\label{code:worldgeneration}
\begin{minted}[
bgcolor=LightGray,
framesep=2mm,
baselinestretch=1.2,
fontsize=\footnotesize,
linenos,
]{csharp}
public void GenerateMap (int x, int z, bool wrapping) {
    Random.State originalRandomState = Random.state;
    if (!useFixedSeed) {
        seed = Random.Range(0, int.MaxValue);
        seed ^= (int)System.DateTime.Now.Ticks;
        seed ^= (int)Time.unscaledTime;
        seed &= int.MaxValue;
    }
    Random.InitState(seed);

    cellCount = x * z;
    overworldGrid.CreateMap(x, z, wrapping);
    if (searchFrontier == null) {
        searchFrontier = new HexCellPriorityQueue();
    }
    for (int i = 0; i < cellCount; i++) {
        overworldGrid.GetCell(i).WaterLevel = waterLevel;
    }
    CreateRegions();
    CreateLand();
    ErodeLand();
    CreateClimate();
    CreateRivers();
    SetTerrainType();
    for (int i = 0; i < cellCount; i++) {
        overworldGrid.GetCell(i).SearchPhase = 0;
        overworldGrid.GetCell(i).MapType = HexMapType.Overworld;
    }

    Random.state = originalRandomState;
}
\end{minted}
\end{listing}
\input{6.prototype/6.4.spielbeginn.tex}
\input{6.prototype/6.5.gameplay.tex}
\subsection{Spielende}
Analog zu \textit{SimCity 2000} gibt es kein direkt vordefiniertes Ende. Der Endzustand des Game Over wird jedoch erreicht, wenn keine Königin mehr verfügbar ist, um die Kolonie zu leiten. Das Spiel ist also theoretisch unendlich lange spielbar, sollte jedoch von der Schwierigkeit so gesteigert werden, dass ein natürliches Ende nach einiger Zeit eintritt. Die Endzustände des Spieles können später noch angepasst werden, eine Idee wäre es, dem Spieler nach Sterben der Königin noch etwas mehr Zeit zu geben um darauf reagieren zu können, sodass eine gerade herangezogene Königin noch die Chance hat zu schlüpfen. Es könnte auch die maximale Jahreszahl begrenzt werden und dem Spieler eine Siegbedingung bereitstellen, in Form eines Überlebens für eine gewisse Zeit.

\subsection{Mechaniken}
Es wurden in vorherigen Sektionen einige Mechaniken und Eigenheiten von älteren und neu erschienen Management Games untersucht und Hypothesen dazu angeführt. Diese Hypothesen und Mechaniken, welche sich als besonders interessant und nutzerfreundlich erweisen, werden nun unter Berücksichtigung der erörterten Vorgänge innerhalb einer Bienenkolonie konkretisiert und ausgefeilt. Außerdem werden die definierten Eigenschaften eines Resource Management Games berücksichtigt und versucht, die untersuchten Gegebenheiten (Ressourcen, Ökonomie, Informationsgehalt und Verwaltungsaspekte) möglichst sinnvoll mit den Vorgängen einer Bienenkolonie zu kombinieren, um daraus ein Spiel beziehungsweise Prototypen zu gestalten, welches sehr stark an die echten Vorgänge einer solchen angelehnt ist. Im folgende referenzierte Hypothesen aus \autoref{table:hypotheses} werden mit [HX] abgekürzt, wobei X für die jeweilige Nummer der Hypothese steht.

\subsubsection{Anweisungsstil}
Wie sich in den Interviews und der Analyse von RimWorld gezeigt hat, lassen sich die Art, auf welche Anweisungen an die jeweiligen Einheiten erteilt werden, in \textit{direktes} und \textit{indirektes} Anweisen beziehungsweise Zuweisen. Während in den bereits vorgestellten Titeln meistens auf eine direkte Anweisung der Einheiten zurückgegriffen wird, stellt RimWorld eine Neuerung dar, welche laut [H15] für das Genre des Colony Managements als durchaus positiv von den Probanden aufgefasst wird. Daher wird dieser Anweisungsstil, mitsamt der Prioritätenliste [H13], welche mittels Grafiken verständlicher gemacht wird [H14]. Diese Grafiken finden sich in \autoref{image:prioritylist} wieder. Es gibt fünf verschiedene Prioritätszustände pro Tätigkeitsbereich pro Biene. Falls die Biene aufgrund ihrer Kaste oder ihres Alters eine bestimmte Tätigkeit nicht machen kann, ist der Button nicht vorhanden. Ansonsten gilt, dass ein grüner Pfeil nach oben bedeutet \textit{Hohe Priorität}, ein gelber Strich bedeutet \textit{Mittlere Priorität}, ein roter Pfeil nach unten bedeutet \textit{Niedrige Priorität}. Falls keine Grafik in dem Button angezeigt wird, steht dies für \textit{Untätig}, womit der Spieler auch entscheiden kann, dass manche Bienen bestimmte Tätigkeiten nicht erledigen, selbst wenn möglich. Wird eine Aktion auf ein Feld angewiesen, wird eine \textit{JobOrder} erstellt, welche alle wichtigen Informationen enthält (vgl. \autoref{code:addjoborder}). Diese JobOrder wird in eine Liste hinzugefügt, welche jedes Frame iteriert wird, und geschaut, ob noch offene Anfragen vorhanden sind. Für jede offene Anfrage wird anhand einer Liste vieler Kriterien eine Biene aus der Liste aller vorhandenen adulten Bienen gesucht (vgl. \autoref{code:assignjobs}). Zuerst werden alle Bienen gesucht, welche keine Anfrage zugewiesen haben, theoretisch die Anfrage ausführen können und ausgewachsen sind. In einer zweiten Iteration werden die nun ausgesuchten Bienen auf Prioritäten geprüft, die Bienen mit der jeweils höchsten Priorität werden in einer Liste gespeichert. Zuletzt werden spezifische Informationen gesucht, wie etwa ein passendes Inventar für die Anfrage, beispielsweise genug Pollen für die Anfrage Pollinate. Letztendlich sollte von dieser dritten Liste die Biene mit dem kürzesten Pfad ausgewählt werden, wobei dieser letzte Schritt aus zeitlichen Gründen noch nicht im Prototyp vorhanden, aber algorithmisch durchaus möglich ist.

\begin{figure}
    \begin{center}
        \includegraphics[width=300px]{0.bilder/prioritylist.png}
    \end{center}
    \caption{Fertig implementierte Prioritätenliste} \label{image:prioritylist}
\end{figure}

\definecolor{LightGray}{gray}{0.9}
\begin{listing}[H]
\caption{Erstellung einer neuen Anweisung}
\label{code:addjoborder}
\begin{minted}[
bgcolor=LightGray,
framesep=2mm,
baselinestretch=1.2,
fontsize=\footnotesize,
linenos,
]{csharp}
public JobOrder AddJobOrder(Bee bee, BeeAction action, HexCell hexCell) {
    JobOrder jobOrder = new JobOrder();
    jobOrder.Action = action;
    jobOrder.AssignedBee = bee;
    jobOrder.Finished = false;
    jobOrder.MapType = hexCell.MapType;
    jobOrder.Cell = hexCell;
    
    jobQueue.Add(jobOrder);
    hexCell.AssignJob(jobOrder);
    hexCell.ShowJobHighlight();
    return jobOrder;
}
\end{minted}
\end{listing}

\definecolor{LightGray}{gray}{0.9}
\begin{listing}[H]
\caption{Zuweisung einer Biene von einer offenen Anweisung}
\label{code:assignjobs}
\begin{minted}[
bgcolor=LightGray,
framesep=2mm,
baselinestretch=1.2,
fontsize=\footnotesize,
linenos,
]{csharp}
private void AssignJobs() {
    for (int i = 0; i < jobQueue.Count; i++) {
        JobOrder jobOrder = jobQueue[i];
        Bee bee = jobOrder.AssignedBee == null ? 
        FindBee(jobOrder) : jobOrder.AssignedBee;
        if (bee) {
            jobOrder.AssignedBee = bee;
            bee.AssignJob(jobOrder);
            this.activeJobs.Add(jobOrder);
            this.jobQueue.Remove(jobOrder);
            bee.Travel(jobOrder.Cell);
        }
    }
}
\end{minted}
\end{listing}

\subsubsection{Ressourcen}
Die im Spiel vorhandenen Ressourcen sind in \autoref{table:resources} aufgelistet. Die Gründe für die Auswahl der Ressourcen sind \textit{Sektion 5.1.1 Nahrung} zu finden. \textit{Honigtau} wird nicht als Ressource implementiert, da diese von lebenden Tieren extrahiert wird, und aufgrund von weiteren benötigten Modellen und Animationen keine weiteren Tiere neben den eigentlichen Bienen implementiert werden. Um das Spiel spannender und etwas komplexer zu gestalten werden verschiedene \textit{Sources}, \textit{Converter} und \textit{Drains} (vgl. \textit{Sektion 1.1.2}) verwendet. Alle involvierten Ressourcen, wie auch baubare Waben, sind in \autoref{image:resourceloop} skizziert. Die Zahlen sind dabei rein experimentell und müssen im Verlauf der Tests gegebenenfalls angepasst werden.

\begin{table}[]
    \centering
    \caption{Verfügbare konkrete Ressourcen}
    \label{table:resources}
    \begin{tabular}{|l|l|}
    \hline
    Nahrung     & Nektar, Honig, Pollen, Royal Jelly, Wasser \\ \hline
    Baumaterial & Bienenwachs                                \\ \hline
    Einheiten   & Königinnen, Drohnen, Arbeitsbienen         \\ \hline
    \end{tabular}
\end{table}

\paragraph{Nektar}
Nektar wird ausschließlich von Blumen extrahiert, welche eine \textit{Source} darstellen. Jedoch ist an jede Blume eine Bedingung geknüpft. Blumen haben ein eigenes Inventar für Nektar und Pollen, welches bei Extraktion durch eine Biene geleert wird und neu regenerieren muss. Eine Biene sollte also frequentiv die Blumen wechseln, um möglichst viele Ressourcen zu sammeln. Nektar hat eine gewisse Lebensdauer und wird nach einiger Zeit schlecht, wodurch eine gewisse Anzahl von Nektar aus dem Spiel entfernt wird. Diese Lebensdauer agiert dadurch als \textit{Drain}.

\paragraph{Pollen}
Die zweite Ressource sind die Pollen, welche ebenfalls von Blumen extrahiert werden. Pollen können unter anderem verwendet werden, um neue Blumen zu pflanzen und spielen daher gerade im Frühling eine große Rolle, aber auch für das Herstellen für Bienenwachs und Gelée Royal. Pollen besitzen eine etwas längere Lebensdauer und können daher über den Winter gelagert werden. Dadurch, dass Pollen mehrere Anwendungsfälle besitzen, muss der Spieler abwägen, für welchen er diese Ressource am ehesten verwendet beziehungsweise wie er die vorhandenen Ressourcen aufteilt.

\paragraph{Honig}
Honig ist eine manuell hergestellte Ressource, welche an einem \textit{Evaporator}, also einer leicht umgebauten Wabe, von einer Biene hergestellt werden kann. Dieser Evaporator stellt, neben den anderen baubaren Waben, einen \textit{Converter} dar. Ein wichtiger Aspekt des Honigs ist der Fakt, dass man mehr als ein Nektar pro Honig verwenden muss, damit der Spieler abwägen muss, ob es die Herstellung wert ist. Dadurch entsteht mehr Komplexität und der Spieler besitzt Entscheidungsfreiheit. So könnten manche Spieler versuchen, lediglich genug Honig für den Winter herzustellen, und andere wiederum jeglichen Nektar direkt in Honig umwandeln. Es entsteht somit eine mögliche \textit{Strategie}.

\paragraph{Bienenwachs}
Entgegen der anderen Ressourcen ist Bienenwachs keine Nahrung. Diese Ressource wird einzig und allein für den Bau neuer Strukturen verwendet. Diese Strukturen werden im Folgenden näher erläutert. Bienenwachs muss an einem \textit{Mixer} hergestellt werden, welcher sowohl Honig als auch Pollen verwendet, um neues Bienenwachs herzustellen. Aus diesem Bienenwachs können dann \textit{Brutwaben, Lagerwaben} und \textit{Refiner} hergestellt werden, wie auch ein neuer Stock, sollte man den gegebenen verlagern wollen.

\paragraph{Gelée Royal}
Eine weitere fundamentale Ressource ist das Gelée Royal oder auch \textit{Royal Jelly}, welches an einem \textit{Refiner} durch Vermischen von Pollen und Honig erzeugt werden kann. Diese Ressource ist wichtig für das Heranziehen einer neuen Königin und wird in die Brutwabe einer \textit{Arbeitsbienenlarve} gegeben, damit diese sich zu einer Königin entwickelt.

\paragraph{Wasser}
Die einzige nicht vom Spieler sammelbare oder lagerbare Ressource ist Wasser. Dieses müssen die Bienen durch Nahrung aufnehmen (Nektar besitzt mehr Wasser als Honig), oder über das Trinken an einem Fluss. Da im Winter keine Ausflüge gestattet sind, ist es wichtig, dass Bienen vor dem Winter genug Wasser zu sich nehmen, um nicht zu verdursten.

\paragraph{Königin, Drohne und Arbeitsbiene}
Diese greifbaren Ressourcen sind das Fundament des Spiels, wobei neue Bienen, analog zu \textit{Sektion 5.1.3 Fortpflanzung}, Kasten-spezifisch herangezogen werden können, mehr dazu im folgenden Abschnitt.

\begin{figure}
    \begin{center}
        \includegraphics[width=300px]{0.bilder/resourceloop.png}
    \end{center}
    \caption{Ressourcenverlauf und mögliche Umwandlungen} \label{image:resourceloop}
\end{figure}

\subsubsection{Jahreszeiten}
Es gibt vier verschiedene Jahreszeiten, \textit{Frühling, Sommer, Herbst} und \textit{Winter}. Alle 30 Tage wechselt die Jahreszeit zur jeweils nächsten. Jede Jahreszeit ist dabei anders und bietet andere, mögliche Events. Jegliche Angaben von Chancen sind dabei rein experimentell und müssen im späteren Verlauf getestet werden. Im Prototypen werden aufgrund der niedrigen Zeitspanne keine Events implementiert sein, jedoch bereits konzipiert für die spätere Implementation.

\paragraph{Frühling} ist die Zeit, in welcher neue Blumen sprießen. Es gibt ein besonderes Event namens \textit{Pollenflug}, wobei besonders viele Blumen sprießen, was es dem Spieler durchaus erleichtern kann, neue Nahrung zu beschaffen. Das Event hat eine Chance von $\frac{1}{30}$ pro Tag zu passieren, und kann pro Jahr maximal ein Mal geschehen.

\paragraph{Sommer} ist die Zeit der Hitze und des \textit{Schwärmens}. Deshalb gibt es zwei Events welche passieren können, das erste ist die \textit{Dürre}, wobei einige Blumen sterben und manche Flüsse austrocknen. Die Chance dafür ist pro Tag $\frac{1}{60}$. Das zweite ist das Schwärmen, welches jedes Jahr erneut passiert, insofern eine Königin vorhanden ist. Dabei wird angegeben, dass die Königin den Stock verlassen wird. Deshalb ist der Spieler dazu gezwungen, eine neue Königin heranzuziehen, da ansonsten keine Königin mehr vorhanden sein wird. Die Königin wird dabei einen kleinen Anteil der Kolonie mitnehmen (mindestens eine \textit{Arbeitsbiene} und \textit{20\%} der Arbeitsbienen). Diese Zahl ist variabel und wird gegebenenfalls angepasst. Der Spieler wird also etwas zurückgesetzt und gezwungen, zu handeln, was die Herausforderung durchaus erschwert. 

\paragraph{Herbst} bringt zwei negative Events mit sich. Es kann jeden Tag mit einer $\frac{1}{60}$ Chance passieren, dass jegliche Blumen weniger Ertrag geben in entweder Pollen oder Nektar. Auch hier wird der Spieler dazu gezwungen, sich anzupassen und sinnvoll darauf zu reagieren. Zudem ist die Chance auf Regen durchaus etwas höher, als sonst in den Jahreszeiten. Als wiederkehrendes Event, analog zum Schwärmen, steht jedes Jahr der \textit{Drohnenkrieg} beziehungsweise die \textit{Drohnenschlacht} an, wobei sämtliche Drohnen aus dem Stock 

\paragraph{Winter} ist die Zeit des Notstands. Die Bienen sind in dieser Zeit nicht ausflugfähig und müssen in ihrem Bienenstock warten, bis der Winter endet. Daher muss darauf geachtet werden, dass genug Honig gelagert ist, damit die Kolonie überleben kann. Sollte Regen während dieser Zeit fallen, ist dieser stattdessen Schnee. Es sterben zu dieser Jahreszeit sämtliche Blumen ab, wodurch, selbst wenn eine Biene rausfliegen würde, es keine Möglichkeit gibt, Nahrung zu beschaffen.


\begin{figure}
    \begin{center}
        \includegraphics[width=100px]{0.bilder/beeinfodraw.PNG}
    \end{center}
    \caption{Skizze des Informationsfensters einer ausgewählten Biene} \label{image:beeinfodraw}
\end{figure}

\begin{figure}
    \begin{center}
        \includegraphics[width=200px]{0.bilder/beeuml.png}
    \end{center}
    \caption{Simplifiziertes UML-Klassendiagramm der Bee.cs} \label{image:beeuml}
\end{figure}

\subsubsection{Bienen}
Die Bienen sind das Herzstück des Spiels und werden analog zu \textit{RimWorld} indirekt gesteuert. Eine Biene hat mehrere Eigenschaften. Sämtliche aufgelisteten Eigenschaften sind visuell dargestellt als Skizze beziehungsweise Mockup in \autoref{image:beeinfodraw}. Die Klasse der \textit{Bee} kann über das leicht vereinfachte UML-Klassendiagramm in \autoref{image:beeuml} eingesehen werden. In dem dargestellten UML-Diagramm sind einige Felder nicht gezeigt, da diese für äußere Zugriffe unwichtig sind. Außerdem sind die meisten vorhandenen Methoden nicht aufgelistet, da die Anzahl dieser das Diagramm aufblähen würden. Die gesamte Klasse umfasst einen Zeilenumfang von \textit{673 Zeilen}.

\paragraph{Kasten}
Es gibt analog zu einer realen Bienenkolonie drei verschiedene Kasten. Jede Biene besitzt eine Variable in Form einer Enum, welche angibt, ob es sich um eine Königin, eine Drohne oder eine Arbeitsbiene handelt. Aufgrund dieser Information werden etliche Entscheidungen getroffen, darunter das gezeigte Modell, wovon es drei verschiedene für jede Kaste gibt, die möglichen ausführbaren Aktionen und die Lebensdauer. Es können kastenspezifisch neue Bienen herangezogen werden. Wie auch in der realen Welt werden für die Arbeitsbienen befruchtete Eier benötigt, für die Königinnen befruchtete Eier und Royal Jelly, und für die Drohnen lediglich unbefruchtete Eier. Diese Mechanik ist in \autoref{image:reproductionloop} schematisch dargestellt.

\begin{figure}
    \begin{center}
        \includegraphics[width=300px]{0.bilder/reproductionloop.png}
    \end{center}
    \caption{Reproduktionsverhalten der Bienen} \label{image:reproductionloop}
\end{figure}

\paragraph{Name}
Jede Biene hat zur besseren Unterscheidung einen zufällig generierten Namen aus einer großen Variation verschiedenster Ethnien. Der Generator entstammt dem Unity Asset Store \cite*[]{asset:namegenerator}.

\paragraph{Alter} 
Das Alter der Biene ist von zentraler Bedeutung. Für die Königin ist es entscheidend einzuschätzen, wie lange diese noch lebt, für die Arbeitsbienen ist es wichtig zu erkennen, welche Arbeiten und Aufgaben diese übernehmen kann. Eine Biene am Ende ihres Lebens ist biologisch nicht mehr in der Lage, die neue Brut zu füttern, und dient lediglich dem Sammeln von Ressourcen.

\paragraph{Hunger} 
Eine Biene hat das Bedürfnis nach Nahrung. Der Nahrungsbalken wird mit der Zeit weniger, weshalb die Biene Nahrung beschaffen muss, jedoch nicht aktiv. Ist Nahrung vorhanden und die Biene erreicht einen Schwellenwert, wird sich diese Biene die Nahrung eigenständig aus dem Lager entnehmen. Ist der Wert der Nahrung zu lange auf 0, stirbt die Biene. Die voreingestellte Reihenfolge der verspeisten Nahrung lautet:

\begin{center}
    Nektar > Honig > Pollen
\end{center} Royal Jelly wird nicht von den Arbeitern angerührt und wird lediglich von der Königin verspeist, genau so wie für das Heranziehen neuer Königinnen zur Larve gegeben. Stetige Bewegung sorgt für schnelleren Hunger. Die Reihenfolge der Nahrung kann vom Spieler angepasst werden, wodurch mehr Entscheidungsfreiheit gegeben ist und mehr Taktik angewandt werden kann. So könnte ein Spieler möglichst viele Pollen sammeln und diese als primäre Nahrungsquelle nutzen. Natürlich müssen diese Mechaniken ausgiebig getestet werden, sodass verschiedene Strategien nicht zu unausgeglichen sind. Diese Änderung, wie auch das Futtern oder Füttern, ist im gegebenen Prototypen nicht implementiert.

\paragraph{Durst} 
Bienen benötigen Flüssigkeit, genauer gesagt Wasser. Dieses wird nicht im Stock gelagert, sondern an Flüssen getrunken. Je mehr sich eine Biene bewegt, desto mehr Flüssigkeit benötigt diese. Sinkt dieser Wert auf 0, wird die Biene nach kurzer Zeit sterben. Die Funktionalität des Trinkens ist im Prototypen noch nicht implementiert, jedoch stirbt eine Biene an Durst, sobald der gegebene Wert auf 0 sinkt.

\paragraph{Samen} 
Diese Eigenschaft hat lediglich eine Königin, und gibt an, wie viel Samen der Drohnen noch vorhanden sind um neue und befruchtete Eier zu legen. Es ist wichtig darauf ein Auge zu haben, da ohne verfügbare Samen weder Arbeitsbienen, noch Königinnen herangezogen werden können. Die Aktion zieht aus dem Pool der vorhandenen in der Königin gespeicherten Samen einen Wert ab, sodass die Anzahl der Aktion durch diese nicht greifbare Ressource limitiert ist.

\paragraph{Inventar} 
Das Inventar besitzt jede Biene und zeigt an, was diese gerade trägt. Getragen werden können Nektar, Pollen, Honig, Royal Jelly und Bienenwachs. Die Menge richtet sich nach der Art des Getragenen. Das Inventar ist ein zentraler Punkt der Ressourcen, da Bienen für Aktionen in den meisten Fällen die benötigte Ressource erst aus dem Lager holen müssen, falls nicht bereits im Inventar vorhanden, und dann diese nicht greifbaren Ressourcen weiterverwenden mittels Converter.

\paragraph{Auftrag} 
Der momentane Auftrag ist ebenfalls Teil der Biene. Dieser zeigt dem Nutzer, was diese Biene gerade vorhat und wieso sie sich zu der bestimmten Stelle bewegt.

\paragraph{Genetische Eigenschaften / Traits} 
Die genetischen Eigenschaften beziehungsweise \textit{Traits} sind immer exakt drei Stück und werden nach [H6] in überschaubarer Anzahl gehalten und so benannt, dass aus dem Namen der Effekt grundlegend ersichtlich ist. Diese können sowohl \textit{positive} als auch \textit{negative} Auswirkungen auf die Effizienz der einzelnen Biene haben, wobei nicht eindeutig erkennbar sein soll, welcher der effektiv beste oder schlechteste Trait ist, der zur Auswahl steht, um [H7] nachzugehen, und Nutzer-eigene Strategien zu fördern. Die möglichen Traits und ihre Wahrscheinlichkeiten sind in \autoref{table:traits} zu finden. Die Traits sind vererbbar, wobei bei der Vererbung eine Chance besteht, dass ein Trait \textit{mutiert} und deshalb in einen anderen, zufälligen, geändert wird. Wird eine Königin von mehreren verschiedenen Drohnen besamt, werden die Erbinformationen gespeichert. Ob der jeweilige Trait nun von der Seite des Vaters oder der Seite der Mutter übernommen wird, ist gleichverteilt 50:50. 

Die Traits, welche speziell auf eine gewisse Art von Kaste gelten, werden im Hintergrund gespeichert und es wird ein neuer Trait gezogen. Dieser wird sich vorgemerkt. Sollte die Arbeitsbienenlarve also zur Königin herangezogen werden, wird der vorgemerkte Trait durch den Erstgezogenen ersetzt. Die Traits sind bereits namentlich implementiert, es werden drei bei der Geburt der Biene zugewiesen. Allerdings haben diese im Prototypen noch keine weiteren Effekte und keine passenden Grafiken. Auch werden die Traits noch nicht genetisch vererbt, sondern zufällig gezogen, was sich im Laufe der Entwicklung noch ändern wird.

\paragraph{Mutation}
Damit eine Diversität entstehen kann, und nicht stetig die gleichen Traits in der Kolonie weitervererbt werden, bedarf es der Möglichkeit einer Mutation. Es besteht eine geringe Chance, dass \textit{nach dem Ziehen} eines vererbten Traits, dieser Trait durch einen völlig anderen ersetzt wird. Diese Mechanik ist in dem Prototypen noch nicht gegeben.
% Please add the following required packages to your document preamble:
% \usepackage{graphicx}
\begin{table}[]
    \centering
    \caption{Mögliche Traits einer Biene}
    \label{table:traits}
    \resizebox{\columnwidth}{!}{%
    \begin{tabular}{|l|l|l|}
    \hline
    Arbeitsfreudig / Arbeitsscheu (Alle)        & Erledigt alle Aufgaben 10\% schneller / langsamer                                        & 12\% \\ \hline
    Sammelfreudig / Sammelscheu (Arbeitsbiene)  & Sammelt Ressourcen 15\% schneller / langsamer                                            & 12\% \\ \hline
    Fingerfertig / Grobmotorisch (Arbeitsbiene) & Stellt neue Ressourcen 10\% schneller / langsamer her.                                   & 12\% \\ \hline
    Sparsam / Freigiebig (Arbeitsbiene)         & Verbraucht beim Füttern etwas weniger / mehr Nahrung.                                    & 12\% \\ \hline
    Empfänglich / Unempfänglich (Königin)       & Benötigt weniger / mehr Drohnen um die Samen zu füllen und hat mehr / weniger Kapazität. & 12\% \\ \hline
    Gesegnet / Verflucht (Alle)                 & Lebt ein längeres / kürzeres Leben.                                                      & 12\% \\ \hline
    Wasserspeicher / Verdampfer (Alle)          & Benötigt weniger / mehr Wasser und verbraucht weniger / mehr.                            & 12\% \\ \hline
    Kleiner / Großer Magen (Alle)               & Benötigt weniger / mehr Nahrung und verbraucht weniger / mehr.                           & 12\% \\ \hline
    Königliche Immunität (Drone)                & Die Drohne ist ausgenommen von der Drohnenschlacht.                                      & 2\%  \\ \hline
    Schillernd (Alle)                           & Die Biene sieht anders aus als gewöhnliche Bienen.                                       & 2\%  \\ \hline
    \end{tabular}%
    }
    \end{table}



\subsubsection{Aktionen}
Es gibt 6 verschiedene, vom Spieler ausführbare, Anordnungen. Diese werden verwendet, um die Bienen indirekt zu steuern und den Stock somit möglichst am Leben zu erhalten. Von den Aktionen sind im Prototyp das Sammeln, das Bestäuben, das Eier legen und das Abbrechen funktional. 

\paragraph{Sammeln} wird verwendet, um von den in der Spielwelt vorhandenen Blumen den Nektar und die Pollen zu extrahieren. Diese Aktion ist lediglich von Arbeitsbienen im späten Alter ausführbar.

\paragraph{Bestäuben} verwendet Pollen, um neue Blumen zu erstellen, welche als weitere Quelle von Nektar und Pollen agieren. Dies kann sinnvoll sein, sollten nicht genug Blumen vorhanden sein, etwa im Frühling nach einem harten Winter. Diese Aktion ist ebenfalls nur von älteren Bienen ausführbar.

\paragraph{Besamen} ist eine Aktion, welche lediglich den Drohnen vorbehalten ist. Mit Auswahl der Königin finden sich nacheinander Drohnen, bis die Königin keine weiteren Samen speichern kann.

\paragraph{Eier legen} wird lediglich von der Königin ausgeführt. Nach Auswahl einer passenden Brutwabe (normale Brutwabe oder königliche Brutwabe), fliegt die Königin zu der jeweiligen Stelle und legt ein Ei ab. Ob dieses befruchtet ist oder nicht hängt davon ab, ob die Königin Samen gespeichert hat.

\paragraph{Abbrechen} wird verwendet, um eine bereits ausgeführte Anordnung zurückzuziehen. Anders als die anderen Aktionen ist hierfür keine Biene nötig.

\paragraph{Zerstören} kann verwendet werden, um vom Spieler gebaute Strukturen wieder zu vernichten. Diese Aktion ist lediglich von mittelalten Arbeitsbienen ausführbar und ist auf den Bienenstock beschränkt.

\subsubsection{Strukturen}
Es gibt, analog zu den Aktionen, 6 verschiedene, baubare Strukturen. Diese dienen in den meisten Fällen als \textit{Converter} und Arbeitsstätte der Bienen, aber auch als Lagerplatz. Sämtliche baubare Strukturen sind \autoref{image:resourceloop} zu entnehmen.

\paragraph{Evaporator} ist ein Converter, um aus Nektar Honig herzustellen, und dient der Erfüllung von [H12]. Dieser wird, wie in \autoref{image:resourceloop} erkennbar, aus Pollen hergestellt. Mit Klick auf den gebauten Evaporator können Aufträge erstellt werden, wodurch festgelegt werden kann, wie viel Honig hergestellt werden soll. Für diese Aufträge müssen Bienen am Evaporator arbeiten.

\paragraph{Mixer} ist ebenfalls ein Converter, erfüllt ebenfalls [H12] und wird verwendet, um aus Honig und Pollen Bienenwachs herzustellen. Zum Bau dieser Struktur wird Honig verwendet, wodurch ein zuvor gebauter Evaporator sinnvoll ist. Das Auftragssystem ist analog zum Evaporator.

\paragraph{Refiner} wird verwendet, um aus Honig und Pollen Royal Jelly herzustellen, wobei für den Bau dieser Struktur Bienenwachs verwendet wird. Dadurch kann es sinnvoll sein, einen Mixer gebaut zu haben. Das Auftragssystem ist analog zu dem Evaporator und dem Mixer und dient ebenfalls der Erfüllung von [H12].

\paragraph{Lagerwaben} sind essenziell zum Speichern und Lagern bestimmter Ressourcen. In diesen Lagerwaben können Nektar, Pollen, Honig, Royal Jelly und Bienenwachs gespeichert werden, wobei eine Lagerwabe nur eine Art von Ressource halten kann, und ebenfalls ein quantitatives Limit für diese Ressource besitzt. Für den Bau einer Lagerwabe wird Bienenwachs verwendet. Diese Lagerstruktur dient als klare, baubare Struktur der Erfüllung von [H9].

\paragraph{Brutwaben} werden für die Fortpflanzung beziehungsweise Eierhaltung der Königin verwendet. Junge Arbeitsbienen werden lediglich angeordnet Nektar oder Honig zu liefern. Wird die Larve zu einer Puppe, muss außerdem von einer Arbeitsbiene die Brutwabe durch Bienenwachs bedeckt werden.

\paragraph{Königliche Brutwaben} sind ähnlich zu den normalen Brutwaben, mit der Ausnahme, dass lediglich befruchtete Eier dort gelegt werden können, und die jungen Arbeitsbienen zu der Larve Royal Jelly geben, falls vorhanden.

\begin{figure}
    \begin{center}
        \includegraphics[width=300px]{0.bilder/metamorphosisimplemented.jpg}
    \end{center}
    \caption{Vier verschiedene Stadien des Bienenmodells von links nach rechts: Ei, Adult, Puppe und Larve} \label{image:metamorphosisimplemented}
\end{figure}

\subsubsection{Brutvorgang}
Um neue Bienen herzustellen werden, wie bereits zuvor erläutert und in \autoref{image:reproductionloop} erkennbar, verschiedene Bienen für verschiedene Resultate benötigt. Wird ein Ei in eine Brutkammer gelegt, beginnt ab diesem Zeitpunkt die Metamorphose, analog zu den in \textit{Sektion 5.1.4 Metamorphose} erläuterten Vorgängen. Das gelegte Ei wird nach einer gegebenen Zeit zu einer Larve, danach zu einer Puppe und anschließend zu einer voll ausgewachsenen Biene. Die Modelle einer einzelnen Einheit sind in \autoref{image:metamorphosisimplemented} zu erkennen. Die Modelle der Bienen, wie auch der verschiedenen Stadien der Metamorphose wurden von \textit{Aaron Zabel} \cite*[]{aaron0zabel} mit der Software \textit{Blender} entworfen.

\paragraph{Stadium 1: Ei} Das Ei liegt nach dem Legen in der Brutwabe. Zu diesem Stadium ist lediglich erkennbar, ob befruchtet oder unbefruchtet. Die Wabe muss ein Mal mit Nahrung, vorzugsweise Nektar oder Honig, gefüllt werden, damit das Ei überlebt. Nach einigen Tagen schlüpft daraus eine Larve und die Nahrung in der Brutwabe ist verbraucht.

\paragraph{Stadium 2: Larve} Die Larve ist das Stadium, in welchem Royal Jelly dazugegeben werden kann. Liegt das Ei in einer Königlichen Brutkammer, wird einmalig Royal Jelly statt gewöhnlicher Nahrung dazugegeben. Damit entwickelt sich eine königliche Puppe statt einer herkömmlichen und die Nahrung wird verbraucht.

\paragraph{Stadium 3: Puppe} Die Puppe benötigt, anders als die anderen Stadien, zusätzlich zur Nahrung noch Bienenwachs, welches die Brutwabe bedeckt. Zuerst muss Nahrung dazugegeben, danach die Wabe bedeckt werden. Sobald die Biene alt genug ist, schlüpft sie aus der Puppe und bricht durch die Wachsschicht. Eine erwachsene Biene ist somit herangereift.

\paragraph{Stadium 4: Ausgewachsen} Das ausgewachsene Stadium ermöglicht sämtliche Aktionen einer Biene. In diesem Stadium befinden sich die meisten Bienen. 

\begin{figure}
    \begin{center}
        \includegraphics[width=200px]{0.bilder/timecontrol.png}
    \end{center}
    \caption{Zeitsteuerung Zustandsdiagramm} \label{image:timecontrol}
\end{figure}

\subsubsection{Zeitsteuerung}
Analog zu RimWorld wird nach [H21] eine Zeitsteuerung zur besseren Kontrolle eingeführt. Zur Auswahl stehen dabei \textit{Pause} beziehungsweise \textit{Fortführen} (engl. \textit{Resume}) als sich abwechselnde Zustände, und weiterhin \textit{Normal} und \textit{Erhöht}, wobei \textit{Pause} die Geschwindigkeit des Spiels auf $g=0$ setzt, \textit{Normal} die Geschwindigkeit auf $g=1$ setzt, und \textit{Erhöht} die Geschwindigkeit auf $g=2$ setzt. Allerdings wird eine intelligentere Logik verwendet, um das Nutzererlebnis zu erhöhen, welche in \autoref{image:timecontrol} veranschaulicht wird. Pausiert der Nutzer das Spiel, während die Geschwindigkeit auf g=2 steht, wird sich dieser Wert gemerkt, und bei Klick auf den nun vorhandenen Resume-Button wiederhergestellt. Analog funktioniert dieser Prozess auch bei g=1.


\subsection{Interface}
Für alle genannten Mechaniken und Informationen muss ein passendes Interface erstellt werden. Der erste Prototyp ist erkennbar in Form eines Mockups in  \autoref{image:interfaceprototype}. Das Interface wird aus einzelnen Komponenten bestehen, welche in der Grafik durch die rot markierten Bereiche erkennbar sind und in folgenden Paragraphen mit (X) referenziert werden, wobei X dem roten Bereich mit der Zahl X in \autoref{image:interfaceprototype} zugeordnet wird.

\begin{figure}
    \begin{center}
        \includegraphics[width=300px]{0.bilder/interfaceprototype.png}
    \end{center}
    \caption{Erster Prototyp des Spielinterfaces} \label{image:interfaceprototype}
\end{figure}

\subsubsection{Karte wechseln}
Um zwischen der Außenwelt und dem inneren des Bienenstocks hin- und herzuwechseln gibt es den Button, welcher in Bereich (1) von \autoref{image:interfaceprototype} dargestellt wird. Befindet sich die Kamera in der Außenwelt, ist auf diesem Knopf ein Bienenstock abgebildet, befindet sich die Kamera im Bienenstock ist dort ein Baum abgebildet, welcher Symbolisch für die Außenwelt steht.

\subsubsection{Action Buttons}
Zur Steuerung der Bienen bedarf es verschiedener auswählbaren Aktionen, welche bereits zuvor erläutert wurden. Es gibt insgesamt 12 verschiedene Aktionen, welche in zwei hexagonale Menüs aufgeteilt sind, zum einen die Tätigkeiten und zum anderen die Strukturen. Diese Menu Buttons sind mit einem zweiten, inneren Hexagon markiert und in Bereich (2) auffindbar. Klickt man auf den jeweiligen Menü-Knopf öffnen sich die jeweils 6 untergeordneten Buttons. Ist währenddessen das andere Menü geöffnet, wird es geschlossen. Es kann somit maximal nur ein Menü gleichzeitig aktiv sein, aber es können beide zeitgleich geschlossen werden. Für den in der Arbeit angefertigten Prototypen werden lediglich die Aktionen \textit{Gather}, \textit{Pollinate}, \textit{Build Storage}, \textit{Lay Eggs} und \textit{Cancel} funktional implementiert. Die ursprüngliche Idee war es, sämtliche Aktionen in ein Menü zu verpacken, wurde dann jedoch aufgrund der schlechteren Übersicht verworfen und aufgeteilt.

\subsubsection{Bee List}
Für einen besseren Überblick der Anzahl verschiedener in der Kolonie vorhandener Kasten wird eine kleine Übersicht dargestellt (3). Zum jeztigen Zeitpunkt werden dort 6 verschiedene Werte angezeigt, je ein Element pro Kaste und je ein Element pro prä-adulter Biene je Kaste. Der Button an der linken Seite öffnet die Liste der Prioritäten, welche bereits zuvor erläutert wurde (vgl. \autoref{image:prioritylist}). Die Grafiken der jeweiligen Kaste oder des Jobs sind noch nicht unterschiedlich dargestellt. 

\subsubsection{Alerts}
Damit der Spieler leichter Überblick über kritische Situationen, wie auch stattfindende Events, haben kann, wird in der Mitte des Bildschirms eine kleine Benachrichtigung visualisiert (4). Ein Alert bleibt für einen kurzen Moment und verschwindet dann wieder, der Spieler wird damit angeregt, aufmerksamer zu spielen und bedient sich der Hypothese [H8]. In dem Prototyp kann mit Drücken der Taste 9 auf der Tastatur für Testzwecke ein kleiner Alert aufgerufen werden. In der späteren Implementierung können sehr einfach neue Alerts ergänzt werden.

\subsubsection{Resource List}
Analog zu RimWorld wird dem Spieler der globale Speicherzustand aller Ressourcen angezeigt (5). Es werden also die Summen aller verschiedenen sammel- und tragbaren Ressourcen angezeigt: \textit{Nectar, Pollen, Wax, Honey} und \textit{Royal Jelly}. Der Button auf der rechten Seite ist im Prototypen noch nicht funktional, soll aber eine Statistik offenbaren, um den Verlauf der Ressourcenstände zu visualisieren. Dies soll dem Spieler etwas mehr Überblick über den Werteverlauf der Kolonie bieten. Da das Lager noch nicht funktional, sondern lediglich dekorativ ist, wird jederzeit bei jeder Ressource eine 0 angezeigt.

\subsubsection{Infofenster}
An der rechten Seite des Bildschirms ist ein Infofenster auffindbar, welches Informationen zum ausgewählten Objekt darstellt. In dem Mockup des Prototypen ist das angezeigte Infofenster jenes, welches bei Auswahl einer Biene gezeigt wird (6). Es werden unter anderem Hunger, Durst, Traits und momentane Aufgabe dargestellt.

\subsubsection{Time Control}
Ebenfalls analog zu RimWorld ist eine Zeitsteuerung implementiert, welche bereits zuvor erläutert wurde. Auf der rechten Seite dieser Steuerung wird das derzeitige Datum angezeigt in dem Format \textit{Year A, Day B of C, Dh}, wobei A das derzeitige Jahr, B der momentane Tag, C die derzeitige Jahreszeit und D die momentane Stunde ist (7). Wie bereits zuvor aufgezeigt gibt es drei verschiedene Grade der Zeit, g=0, g=1 und g=2.

\subsubsection{Start Screen}
Das Interface, welches dem Spieler als allererstes gezeigt wird, ist der Start Screen, welcher für den Anfang sehr simpel gehalten ist nach Hypothese [H1]. Es werden lediglich zwei Buttons zum Erstellen und Laden eines Spielstands angezeigt, ein Hintergrund des Spiels und eine Überschrift mit dem Titel des Spiels \autoref{image:startscreen}. Das Laden und Speichern eines Spielstands sind im Prototyp noch nicht implementiert, die Fundamente dafür sind jedoch bereitgestellt.

\begin{figure}
    \begin{center}
        \includegraphics[width=300px]{0.bilder/startscreen.png}
    \end{center}
    \caption{Start Screen des Spiels} \label{image:startscreen}
\end{figure}

\subsubsection{Spielwelt}
Der hellgraue Bereich in \autoref{image:interfaceprototype} stellt die interaktive Spielwelt dar, in welcher die Aktionen angegeben werden können und die Einheiten sich dementsprechend verhalten.
\subsection{Unit Tests}
Um die Stabilität der Anwendung im Allgemeinen zu gewährleisten, und die Zuverlässigkeit der einzelnen Funktionen sicherzustellen, werden, auch im Bereich des Game Development, sogenannte \textit{Unit Tests} verwendet. Es gibt dabei mehrere Vorteile dieser Tests. Man kann beispielsweise mit diesen Tests Szenarien abdecken, welche nur sehr selten im produktiven Anwendungsfall auftreten würden, sogenannte \textit{Corner Cases}. Außerdem ist die Chance deutlich geringer, dass bereits aufgetretene Bugs, welche durch Unit Tests bereits behoben wurden, erneut auftreten. Der größte Nachteil dieser Tests ist der Fakt, dass deutlich mehr Code geschrieben werden muss, was sehr viel Zeit einfordert \cite*[]{unittests}. In einem industriellen Umfeld und bei der Entwicklung einer marktreifen Anwendung ist dieser Schritt allerdings unerlässlich. Es ist daher sinnvoll, auch im weiteren Verlauf, Unit Tests für die gesamte Anwendung nachträglich hinzuzufügen. Aufgrund der kurzen Zeitspanne und dem bereits großen Umfang des Quellcodes wird in dieser Arbeit bewusst auf diese Art des Testings verzichtet.
