\subsection{User Testing}
Für den Abschluss der Implementierung des Prototypen wird ein letzter Test durchgeführt in Form eines User Testing beziehungsweise Produkttests. Der zuvor entwickelte Prototyp mit dem Arbeitstitel \textit{HexaBees} wird in einer letzten Iteration von einem ausgewählten Probanden getestet, wobei sich jegliche Eindrücke, Ideen und Probleme bei der Auseinandersetzung notiert werden, um in späteren Implementationszyklen eingebunden zu werden. Da der zuvor erarbeite Gameplay Loop nicht vollständig im Prototyp vorhanden ist, wird der Gameplay Loop lediglich theoretisch erörtert und das Feedback dazu notiert. Selbstverständlich wäre es interessant, die Eindrücke und Ideen von mehreren Personen aufzugreifen, wird der Einfachheit halber und aus zeitlichen Gründen nicht auf mehrere Personen erweitert.

\subsubsection{Versuchsaufbau}
Analog zu den Interviews stehen erneut die Fragen und Beobachtung im Vordergrund. Jedoch ist nun die Versuchszeit auf 15 Minuten heruntergesetzt, da sich der Prototyp in einem recht frühen Stadium befindet. Das Transkript zu dem Produkttest befindet sich erneut im Anhang. Anders als im Interview wird der Produkttest auf zwei Phasen begrenzt, die \textit{Einleitung} und der \textit{Hauptteil}. In der Einleitung wird der Versuchsaufbau kurz erläutert, während im Hauptteil Fragen zum Prototypen gestellt und Beobachtungen notiert werden.

\paragraph{Hauptteil}
\begin{itemize}
    \item F1: Wie findest du das Farbschema?
    \item F2: Ist das User Interface übersichtlich?
    \item F3: Findest du die Grafiken ersichtlich und hilfreich?
    \item F4: Ist die Zeit gut erkennbar?
    \item F5: Ist der Sinn der Zeitsteuerung klar?
    \item F6: Wie findest du die generierte Welt?
    \item F7: Findest du es sinnvoll, dass Bienenstock und Außenwelt getrennt sind?
    \item F8: Wie findest du die Auswahl der Ressourcen?
    \item F9: Was hältst du von dem Gameplay Loop?
    \item F10: Denkst du, das voll implementierte Spiel hätte einen gewissen Faktor an Spaß und eine Langzeitmotivation?
\end{itemize}

\subsubsection{Proband}
Der ausgewählte Proband war bereits im zuvor durchgeführten Interview als Proband aufgetreten und hat daher mit absoluter Sicherheit Kontakt mit Resource Management Games gehabt. Da mit den Hypothesen der Prototyp erarbeitet wurde, und damit ein großer Fokus auf Einsteigerfreundlichkeit gelegt wurde, empfiehlt es sich, auf Proband A zurückzugreifen, welcher tendenziell weniger Erfahrung mit Videospielen im Allgemeinen vorweist, als die anderen beiden Probanden.

