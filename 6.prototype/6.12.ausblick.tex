\subsection{Ausblick}
Das Projekt wird in Zukunft fortgesetzt. Der Grundstein wurde gelegt und scheint, laut Ergebnissen, auch vielversprechend zu sein. Neben den bereits konzipierten Mechaniken und Elementen, welche mit Sicherheit implementiert werden, gibt es weitere Ideen, die erörtert werden müssen. Eine solche Idee wäre der Mensch als natürlicher Feind der Bienen. Es könnten zu Beginn des Spiels einige Kacheln, je nach Größe der Gesamtkarte, mit kleinen Dörfern belegt sein. Über die Zeit breiten sich diese Dörfer dann aus auf vorzugsweise zufällige Nachbarkacheln aus. Ist eine Kachel mit einem Dorf vollständig umringt von Kacheln mit Dörfern, wird aus dem Dorf eine Stadt. Dörfer und Städte beherbergen Menschen, welche das Klima beeinflussen, wodurch mit der Zeit die Temperatur (falls implementiert), wie auch der Meeresspiegel, die Flüsse und die Vegetation verändert wird. Außerdem belegen die Dörfer mehr und mehr Kacheln, welche Blumen zerstören und das Erstellen neuer Blumen unmöglich machen. 