\subsection{Unit Tests}
Um die Stabilität der Anwendung im Allgemeinen zu gewährleisten, und die Zuverlässigkeit der einzelnen Funktionen sicherzustellen, werden, auch im Bereich des Game Development, sogenannte \textit{Unit Tests} verwendet. Es gibt dabei mehrere Vorteile dieser Tests. Man kann beispielsweise mit diesen Tests Szenarien abdecken, welche nur sehr selten im produktiven Anwendungsfall auftreten würden, sogenannte \textit{Corner Cases}. Außerdem ist die Chance deutlich geringer, dass bereits aufgetretene Bugs, welche durch Unit Tests bereits behoben wurden, erneut auftreten. Der größte Nachteil dieser Tests ist der Fakt, dass deutlich mehr Code geschrieben werden muss, was sehr viel Zeit einfordert \cite*[]{unittests}. In einem industriellen Umfeld und bei der Entwicklung einer marktreifen Anwendung ist dieser Schritt allerdings unerlässlich. Es ist daher sinnvoll, auch im weiteren Verlauf, Unit Tests für die gesamte Anwendung nachträglich hinzuzufügen. Aufgrund der kurzen Zeitspanne und dem bereits großen Umfang des Quellcodes wird in dieser Arbeit bewusst auf diese Art des Testings verzichtet.