\subsection{Abschluss}
Um die Implementierung des Prototypen abzuschließen werden die erarbeiteten Hypothesen angeschaut und erörtert, inwiefern diese umgesetzt werden konnten. Als Fundament der wissenschaftlich erarbeiteten Grundlage der Konzeption des Spiels ist dieser Schritt unerlässlich und 

\paragraph*{H1}
Das Startmenü im Prototypen besitzt lediglich zwei Buttons zum Erstellen und Laden eines Spielstandes. Der Button zum Laden ist momentan noch nicht funktionsfähig, wird in naher Zukunft aber implementiert. 

\paragraph*{H2}
Die Steuerung oberhalb der Karte anzuzeigen ist scheinbar nicht weiter nötig wie der Produkttest gezeigt hat. Sollte die Steuerung komplexer werden, könnte diese Hypothese erneut aufgegriffen werden.

\paragraph*{H3}
Für den Moment sind noch keine optionalen Inhalte implementiert, aber sollten optionale Umgebungsinhalte für eine höhere Schwierigkeit eingefügt werden, beispielsweise die im Produkttest aufgezeigte Idee der Bären als Fressfeinde, sollte dieser Schritt umgesetzt werden.

\paragraph*{H4}
Dieser Aspekt sollte unbedingt im Laufe der Zeit gegeben sein. Je schneller verschiedene Sprachen unterstützt werden, desto einfacher werden zukünftige Iterationen, da weniger nachimplementiert werden muss, und bei jedem neuen Wort oder Begriff die Übersetzung direkt hinzugefügt werden kann statt nachträglich. Dieser Schritt der Inklusion ist essenziell um eine breitere Masse ansprechen zu können und daher auch kommerziell interessant.

\paragraph*{H5}
Der Plan im Gameplay Loop ist es, den Bienenstock zufällig zu platzieren, sodass der Spieler mal bessere und mal schlechtere Startbedingungen hat. Damit ist dieser Schritt zwar momentan noch nicht umgesetzt, aber im Design berücksichtigt.

\paragraph*{H6}
Zur Zeit des Prototypen gibt es 18 theoretisch implementierte Traits, welche in \autoref{table:traits} zu finden sind. Allerdings haben diese noch nicht die richtigen Grafiken zugeordnet und noch keine Effekte auf das Spielgeschehen, werden jedoch bereits zufällig zugeordnet.

\paragraph*{H7}
Da die Traits noch nicht vollends implementiert sind, ist diese Hypothese noch nicht vollends erfüllt. Jedoch ist diese Hypothese im Design vorgemerkt und soll keine direkte Entscheidung über beste oder schlechteste Traits zulassen.

\paragraph*{H8}
Diese Hypothese ist mit dem Konzept des Alerts umgesetzt wurde, wenn auch nicht in Fülle angewandt.

\paragraph*{H9}
Durch das Konzept der Storage als baubare, kachel-belegende Struktur mittels Aktionen und Anweisungen ist diese Hypothese umgesetzt worden.

\paragraph*{H10}
Das User Testing hat gezeigt, dass die gewählte Stelle der Uhrzeit und des Datums gut erkennbar sind.

\paragraph*{H11}
Diese Idee könnte nach wie vor umgesetzt werden, würde die Komplexität jedoch stark erhöhen. Eine mögliche Umsetzung der Temperatur könnte eine Jahreszeitenbedingte Außentemperatur mit sich bringen, und eine Mindestanzahl benötigter Bienen im Bienenstock, um die Temperatur hoch genug für das Überleben zu halten. Dieses System wird in naher Zukunft jedoch voraussichtlich nicht implementiert.

\paragraph*{H12}
Diese Hypothese wurde versucht im Design zu berücksichtigen, sodass sowohl die Ressourcen, als auch die Converter und die Schritte der Ressourcenumwandlung intuitiv sind.

\paragraph*{H13}
Das in RimWorld gezeigte Prioritätensystem wurde funktionsfähig implementiert und stellt einen festen Kernbestandteil des Spiels dar.

\paragraph*{H14}
Die verschiedenen Stufen der jeweiligen Prioritäten wurden grafisch umgesetzt, womit die Hypothese in das Spielgeschehen übernommen wurde.

\paragraph*{H15}
Das indirekte Anweisen ist implementiert worden mittels der im Quellcode ersichtlichen JobQueue und der JobOrder. 

\paragraph*{H16}
Die Steuerung scheint, laut User Testing, sehr intuitiv zu sein, wodurch diese Hypothese unmittelbar in das Spiel integriert wurde.

\paragraph*{H17}
Durch die zufällig generierte Karte, und den im Design vorgemerkten zufälligen Startpunkt des Bienenstocks, wie auch den verschiedenen Traits, ist einiges an Zufall im Spiel enthalten. Dadurch wird versucht, diese Hypothese weiter zu integrieren und damit die replayability zu steigern.

\paragraph*{H18}
Dieser Aspekt wurde stark berücksichtigt, wodurch in jedem Quadranten des User-Interface mehrere Icons zu finden sind, wie auch in der Prioritätenliste. Es zeigt sich, dass eventuell Fehlinterpretationen auftreten können, welche jedoch mit einem Tutorial sehr leicht beseitigt werden können.

\paragraph*{H19}
Diese Hypothese konnte durch den kleinen Umfang des Prototypen nicht in das Spiel integriert werden, ist im Design jedoch vorgemerkt.

\paragraph*{H20}
Auch diese Hypothese konnte aufgrund der kurzen Zeitspanne noch nicht in das Spiel integriert werden, ist jedoch unerlässlich und steht neben der Integration verschiedener Sprachen weit oben auf der Liste der zukünftigen Iterationen.

\paragraph*{H21}
Das User-Testing hat gezeigt, dass die Zeitsteuerung ein durchaus gut verstandenes Konzept darstellt und die Kontrolle des Nutzers verstärkt. Es ergeben sich daraus keinerlei Nachteile für den Spieler und ist damit ein bleibendes Element des Spiels.