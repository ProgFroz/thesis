
\begin{figure}
    \begin{center}
        \includegraphics[width=400px]{0.bilder/interviewphases.png}
    \end{center}
    \caption{Die Vier Phasen des Interviews} \label{image:interviewphases}
\end{figure}
\subsection{Interviews}
Für die Ermittlung gut funktionierender Mechaniken und Konzepte wird auf die Methode der nutzungsorientierten Gestaltung des kontextuellen Interviews zurückgegriffen \cite*[]{holtzblatt_beyer_1997}. Dazu wird ein Leitfadeninterview angefertigt, um qualitative Daten von den Teilnehmern zu extrahieren und daraus Hypothesen anzufertigen \cite*[]{baur_blasius}. Nachdem die Hypothesen erarbeitet wurden, werden diese mittels einer Umfrage in einem breiten Umfang überprüft. Es werden drei Personen ausgewählt, wobei die Erfahrung aller Personen in Bezug auf das Spiel und das Genre generell variiert. Diese Personen, im Folgenden auch \textit{Probanden} genannt, werden in Einzelgesprächen dazu gebeten, dass Spiel eine Stunde lang zu spielen. Einzelgespräche beziehungsweise -interviews sind Gruppengesprächen vorzuziehen, da die einzelnen Probanden einen entspannteren Zeitplan haben, sich gänzlich auf das Spiel und die Fragen fokussieren können und keine Gefahr laufen, abgelenkt zu werden oder mit anderen Probanden in Gespräche zu gelangen \cite*[]{lankoski_bjork}. Das Interview gliedert sich in vier Phasen (vgl. \autoref{image:interviewphases}).

\subsubsection{Versuchsaufbau}
Im Zentrum des Interviews steht, neben den Einleitungs- und Abschlussfragen, das Videospiel \textit{RimWorld}, welches für eine Stunde lang von den Probanden gespielt wird. Das Spiel läuft zu dem Zeitpunkt der Interviews auf der Version \textit{1.3.3387}, ist auf die englische Sprache eingestellt und wird über die Online-Vertriebsplattform für Computerspiele \textit{Steam} ausgeführt.
Das Interview ist, wie bereits erläutert, in vier Phasen gegliedert und wird in der Wohnung des Interviewers durchgeführt. Wichtig ist, dass alle Probanden leicht andere Qualifikationen besitzen, sodass ein breites Bild über den Zustand des untersuchten Spiels erstellt werden kann. Es ist fundamental zu erkennen, wie ein Spiel von Kennern, wie auch von Neulingen empfunden wird. Jedes Interview wird dabei mit einem Mikrofon aufgenommen und anschließend transkribiert. Die Transkripte sind wörtlich und nicht lautsprachlich angefertigt. Dies bietet den einfachen Vorteil, dass die später aufgestellten Hypothesen mit Textbelegen der Aussagen der jeweiligen Teilnehmer aufgestellt beziehungsweise belegt werden können. Allerdings werden die Namen der Teilnehmer nicht genannt und stattdessen durch ein Pseudonym ersetzt \cite*[S.97]{lankoski_bjork}. Alle angefertigten Transkripte sind im Anhang auffindbar, wovon für den kommenden Abschnitt \hyperref[transcript:A]{Transkript A}, \hyperref[transcript:B]{Transkript B} und \hyperref[transcript:C]{Transkript C} relevant sind.
 
\newpage
\paragraph {Einleitung}
Um die erhobenen Daten richtig kategorisieren zu können, werden als erstes grundlegende Daten zu der Person an sich erfasst:
\begin{itemize}
    \item F1: Wie alt bist du?
    \item F2: Nach welchem Geschlecht identifizierst du dich?
\end{itemize}

\paragraph{Aufwärmen}
Im zweiten Schritt des Interviews werden etwas spezifischere Fragen zu der Erfahrung und dem Verhalten bezüglich Videospielen gestellt:
\begin{itemize}
    \item F3: Hast du bereits Erfahrung in Videospielen?
    \item F4: Hast du bereits Erfahrung in Resource Management Games?
    \item F5: Hast du RimWorld bereits zuvor gespielt?
\end{itemize}

\paragraph{Beobachtung}
Die Phase der Beobachtung beschreibt die Spielphase. Der Proband wird dazu aufgefordert, das Videospiel \textit{RimWorld} zu spielen. Es ist von fundamentaler Wichtigkeit für diese Phase, das die beobachtende Person keinerlei Versuch unternimmt, sich in das Spielgeschehen oder das Verhalten der beobachteten Person einmischt. Es wird auch unterlassen, begangene Fehler aufzuklären oder in irgend einer anderen Art und Weise zu helfen, falls nicht anders möglich. Außerdem ist es wichtig zu erwähnen, dass nicht alle Fragen zwangsweise gestellt werden. Diese Fragen dienen lediglich als Katalog möglicher Fragen, die situativ gestellt werden können.

\begin{itemize}
    \item F6: Welche Emotion wird gerade verspürt?
    \item F7: Ist das Spiel für den Probanden immersiv?
    \item F8: Hat der Proband Verlustängste bezüglich der Kolonisten?
    \item F9: Wieso machst du \_\_\_?
    \item F10: Weißt du, was du nun tun kannst?
    \item F11: Was ist dein nächstes Ziel?
    \item F12: Verwendet der Proband die Prioritätenliste?
    \item F13: Kommt der Proband gut mit der indirekten Anweisungsmechanik zurecht?
\end{itemize}

\paragraph{Abschluss}
In der Abschlussphase wird der Proband gebeten, eine eigene Meinung abzugeben. Außerdem können Fehler aufgeklärt werden und inhaltliche Fragen beantwortet werden, ohne dass sie nun das Verhalten im Spiel verzerren.

\begin{itemize}
    \item F14: Was hast du als gut empfunden?
    \item F15: Was hast du als schlecht empfunden?
    \item F16: Wie findest du das indirekte Anweisen?
    \item F17: Was sagst du zu der Ressourcenvielfalt?
    \item F18: Was hältst du von dem Prioritätensystem?
    \item F19: Wie findest du die vielen zufällig generierten Umstände?
\end{itemize}

\subsubsection{Probanden}
Für die Interviews werden exakt drei Personen ausgewählt, welche sich allesamt in ihrer Erfahrung in Videospielen und auch in ihrer Erfahrung in RimWorld unterscheiden. Es ist damit zu erhoffen, dass ein möglichst breites Spektrum an Informationen extrahiert werden kann. Da Proband C sich zu dem Zeitpunkt des Interviews nicht vor Ort befand, wurde mit dieser Person stattdessen über Discord ein Meeting gehalten, in welchem der Bildschirm des Probanden dem Interviewer geteilt wurde. Der Prozess lief unwesentlich unkomplizierter ab, da beide Personen bereits vertraut mit der Software waren. 

% Please add the following required packages to your document preamble:
% \usepackage{graphicx}
\begin{table}[]
    \centering
    \caption{Jeweilige Antworten der Probanden auf die Einleitungsfragen}
    \label{table:interview}
    \begin{tabular}{|l|c|c|c|}
    \hline
                                  & Proband A & Proband B & Proband C \\ \hline
    Alter                         & 19        & 23        & 24        \\ \hline
    Geschlecht                    & \female        & \male         & \male        \\ \hline
    Erfahrung Videospiele         & $\upchi$      & \checkmark        & \checkmark        \\ \hline
    Erfahrung Resource Management & $\upchi$      & \checkmark        & \checkmark       \\ \hline
    Erfahrung RimWorld            & $\upchi$      & $\upchi$      & \checkmark       \\ \hline
    \end{tabular}
    \end{table}

\subsubsection{Ablauf}
Jegliches Interview beginnt mit den Einleitungsfragen, wobei die Antworten der Probanden zusammengefasst in \autoref{table:interview} zu finden sind. Die Altersspanne der Teilnehmer liegt zwischen 19 und 24 Jahren, mit einem Altersdurchschnitt von 22 Jahren. Zwei der drei Probanden sind männlich, ein Teilnehmer ist weiblich. Die beiden männlichen Teilnehmer weisen beide bereits Erfahrung in Videospielen und Resource Management Games auf, wobei lediglich einer der beiden männlichen Teilnehmer bereits RimWorld zuvor gespielt hat. Alle Teilnehmer wurden darüber in Kenntnis gesetzt, dass ihre Stimmen aufgezeichnet werden, und waren damit einverstanden. Die im Folgenden aufgestellten Hypothesen werden mit [HX] gekennzeichnet, wobei das X eine fortlaufende Nummerierung der jeweiligen Hypothesen darstellt. Die Referenzen auf die Transkripte werden mit [X, Z.Y] dargestellt, wobei X in diesem Fall das Pseudonym des jeweiligen Probanden ist (A, B, C), und Y die jeweilige Zeile in dem gegebenen Transkript.

Zu Beginn fällt auf, dass der Startbildschirm des Spiels sehr viele Buttons enthält, und die Option \glqq New Colony\grqq für Neulinge nicht unbedingt eindeutig assoziiert wird mit dem Beginn eines neuen Spiels [A, Z.27-30]. Es wäre daher sinnvoll, die Anzahl der Knöpfe auf ein Minimum zu reduzieren [H1]. Die Steuerung und die einzelnen Symbole auf der generierten Weltkarte sind nicht eindeutig assoziierbar [A, Z.39-48][B, Z.38-54], eine gewisse, gut sichtbare Erklärung oberhalb der Weltkugel wäre eine hilfreiche Ergänzung [H2]. Für spezielle, stark herausfordernde Gegner oder Umgebungseigenschaften empfiehlt sich die Möglichkeit, diese auch ausstellen zu können [H3]. Das gibt dem Spieler mehr Kontrolle über mögliche Frustration [C, Z.37-40]. Außerdem fällt auf, dass, gerade für Neulinge, die englische Sprache in Videospielen etwas speziell sein kann. Diese Spiele enthalten Begriffe, welche im schulischen oder alltäglichen Kontext eher weniger auftreten und daher erst gelernt werden müssen. Es ist daher sinnvoll, möglichst viele Sprachen aufgrund der höheren Inklusion und den dadurch niedrigeren Frust zu implementieren [H4][A, Z.49]. Die Auswahl der Startkachel hängt scheinbar sehr stark mit der Erfahrung des Probanden zusammen, Proband A wählt den Startpunkt zufällig [A, Z.41-48], um möglichst schnell das eigentliche Spiel zu beginnen, während Proband B den Startpunkt abhängig von den benachbarten Fraktionen macht, da diese als positiv assoziiert werden [B, Z.56-59]. Proband C ist der einzige Proband, welcher sich die Eigenschaften der einzelnen Kachel anschaut und basierend auf konkreteren Informationen die Entscheidung fällt, darunter die Anbauzeit und die Terrain-Beschaffenheit [C, Z.43-44]. Dieser Lernprozess und die damit verbundene, gewonnene Erfahrung, ist nicht schlecht. Es lässt Raum für Verbesserung und Lerneffekte. Dem Spieler sollte daher nicht mitgeteilt werden, welche Kacheln sinnvoller wären, als andere [H5]. In der Auswahl der Kolonisten fällt auf, dass selbst Spieler, welche bereits Erfahrung in anderen Spielen des Genres haben, leicht überfordert mit der Anzahl und Diversität der Eigenschaften der Kolonisten haben können [B, Z.74-77]. Für den Umfang von RimWorld ergibt diese Entscheidung durchaus Sinn, jedoch ist wichtig zu erkennen, das hier gegebenenfalls unnötige Komplexität hinzugefügt wird, welche Spieler frustrieren kann. Daher sollten die Traits auf eine kleinere Menge reduziert werden und intuitiv benannt und gestalten sein, damit sowohl neue, als auch erfahrene Spieler dieses System als gut empfinden [H6]. Mit der Zeit entstehen eigene Taktiken, welche Traits gut und welche als schlecht empfunden werden [B, Z.67-71][C, Z.46-53], was eine durchaus positive Entwicklung ist. Es sollte daher nicht klar erkennbar sein, welche Traits die besten, und welche die schlechtesten sind, der Spieler sollte dies über Zeit und durch mehrfaches Ausprobieren lernen [H7]. Die in \autoref{image:rimworldui} in Bereich (5) dargestellten Hinweise auf den Zustand der Kolonie erweisen sich als sehr hilfreich, da jeder der drei Probanden diese als Informationsquelle verwendet [B, Z.82]. Es könnte daher hilfreich sein, kleine, nicht zu viel verratende, Informationen über negative Eigenschaften an den Spieler preiszugeben [H8]. Somit wäre der Spieler nicht komplett auf sich alleine gestellt, aber dennoch nicht an die Hand genommen, sodass ein Lernprozess stattfinden kann. Das Lagersystem in RimWorld scheint für den Großteil der Probanden nicht ersichtlich zu sein, Proband A findet keine Möglichkeit zu lagern, während Proband B erst nach einer sprachlichen Klärung des Begriffes \glqq Stockpile\grqq\; die Lagermechanik beginnt zu begreifen [B, Z.89-92]. Statt einem versteckten Lagersystem sollte das Lager deutlich ersichtlicher dem Spieler präsentiert werden, in Form einer baubaren Struktur, analog zu einer Kiste oder einem Ablageplatz [H9]. Eine weitere nicht ersichtliche Information ist die momentane Uhrzeit (vgl. \autoref{image:rimworldui}, Bereich (7)), welche zentraler dargestellt werden sollte [H10][B, Z.95-98]. Die Mechanik der Temperatur scheint für Proband B interessant zu sein [B, Z.98], wodurch eine analoge Mechanik eventuell für etwas mehr Komplexität sorgen könnte, ohne dabei mehr Erklärungsbedarf zu kreieren [H11]. Alle drei Probanden verstehen, dass Nahrung in irgendeiner Weise zubereitet werden sollte [A, Z.106-111][B, Z.107-111][C, Z.72-73], wodurch sinnvolle Converter-Mechaniken durchaus intuitiv sein können. Daher sollten Converter an einigen Stellen verwendet werden, um die Komplexität etwas zu erhöhen, ohne das Unverständnis des Spielers zu fördern [H12]. Es zeigt sich im Verlauf, dass die Steuerung der Kolonisten sehr unintuitiv sein kann, falls nicht anders bereits bekannt [A, Z.72]. Die Prioritätenliste, mit welcher die indirekten Anweisungen der Kolonisten optimiert werden können, wird von dem Großteil der Probanden nicht verwendet und teilweise auch falsch verstanden [A, Z.85-90][B, Z.127-128]. Das System der Prioritäten scheint positiven Anklang zu finden [A, Z.135][B, Z.155][C, Z.130-132], wodurch ein analoges System unbedingt implementiert werden sollte [H13], allerdings sollten die Nummern gegebenenfalls durch Symbole oder leichter zu verstehende Darstellungsmöglichkeiten ersetzt werden, welche keine Einbuße in der Komplexität darstellen, aber das Verständnis für neuere Spieler durchaus fördern können [H14]. Auch das indirekte Anweisen der Kolonisten wird größtenteils als positiv empfunden, und für den Anwendungsfall von RimWorld dem direkten Anweisen, wie zu finden in beispielsweise \textit{Age of Empires}, vorgezogen [A, Z.127][B, Z.155][C, Z.122-124]. Während dieses System bei neueren Spielern eher ein Gefühl von Machtlosigkeit auslöst [A, Z.129], schlägt diese Empfindung bei steigender Erfahrung in ein Gefühl von mehr Kontrolle und Vorausplanung [B, Z.160-161], und letztendlich in das positive Gefühl des Beobachtens eines eigenen \glqq Ameisenstamms\grqq [C, Z.123-124] um. Daher wird dieses System auf lange Sicht einen positiveren Effekt bringen, als die alternative des direkten Anweisens, und wird daher vorzugsweise implementiert [H15]. Allerdings sollte es Hinweise auf die Steuerung geben, um anfänglichen Frust zu vermeiden, und neuere Spieler direkt abzuholen [H16]. Es zeigt sich außerdem, dass die vielen, zufällig generierten Umstände, beginnend bei der Weltkugel, der Startkachel und den Kolonisten, tendenziell eher als positiv aufgefasst werden, auch wenn Anfangs etwas unübersichtlich [A, Z.139-134][B, Z.173-175][C, 133-136]. Laut Probanden erhöhe diese Zufälligkeit in vielerlei Hinsicht die replayability des Spiels. Es empfiehlt sich daher, eine ähnliche Struktur zu übernehmen, und eine zufällig generierte Karte, wie auch zufällig generierte Einheiten mit jeweils eigenen, distinkten Eigenschaften bereitzustellen [H17]. Für alle Probanden scheint das User-Interface in einigen Punkten undurchsichtig und unübersichtlich. Proband A empfindet die Menge an Text als sehr überwältigend und problematisch [A, Z.57], wodurch es sinnvoll sein könnte, einige Informationen eher als Bilder beziehungsweise Icons darzustellen, statt als Text [H18]. Proband B hat eher Probleme mit der Anordnung der dargestellten Informationen, darunter die bereits erwähnte Uhrzeit. Solche zentralen Elemente sollten daher sichtbarer dargestellt werden ([H10]). Eine frustrierende Eigenschaft des Spiels ist die teilweise zu stark anziehende Schwierigkeit, welche vermieden werden sollte bei der Implementierung des Prototypen. Die Schwierigkeit sollte schwer genug sein, damit das Gefühl einer Herausforderung entsteht, aber nicht zu schwer, sodass der Spieler sich teilweise auch zurücklehnen kann, und der Kolonie beim Arbeiten zuschauen kann, ohne stetig gezwungen zu sein, Input zu geben [H19][C, Z.114-119]. Letztendlich ist ein Tutorial für das Erlernen des Spiels unerlässlich aufgrund der vielen gegebenen Informationen [A, Z.121-124][B, Z.143-145]. Es gibt dabei verschiedene Möglichkeiten, ein Tutorial zu implementieren, aber dem Spieler sollte jederzeit eine Art Hilfestellung zur Verfügung stehen, damit kein Gefühl von Stagnation oder Frustration entsteht aufgrund fehlender Informationen [H20]. Zuletzt scheint die Kontrolle über die Geschwindigkeit des Spiels als essenzieller Bestandteil, sowohl Proband B [B, Z.78-81] als auch Proband C greifen darauf häufig zurück, lediglich Proband A sieht darin keinen Nutzen [A, Z.77-79], wobei dieser Nutzen mit der Erfahrung in Videospielen generell erkennbar wird. Demnach sollte eine solche Zeitkontrolle ebenfalls implementiert werden, um die Kontrolle des Nutzers zu erhöhen [H21]. 

Die hier aufgestellten Hypothesen sind zur besseren Übersicht und für spätere Referenzen in \autoref{table:hypotheses} aufgelistet.
 

\begin{table}[]
    \centering
    \caption{Übersicht aller aufgestellten Hypothesen der durchgeführten Interviews}
    \label{table:hypotheses}
    \begin{tabular}{|l|l|}
    \hline
    H1 & Die Auswahlmöglichkeiten im Startmenü auf das Nötigste reduzieren         \\ \hline
    H2 & Steuerung und Optionen oberhalb der Weltkugel anzeigen                    \\ \hline
    H3 & Herausfordernde Umgebungseigenschaften an- und ausschalten können         \\ \hline
    H4 & Mehrere Sprachen unterstützen für eine höhere Inklusion und weniger Frust \\ \hline
    H5 & Keine Hilfe bei der Auswahl der Startkachel                               \\ \hline
    H6 & Traits intuitiv und Anzahl gering halten                               \\ \hline
    H7 & Traits nicht eindeutig im Nutzen vergleichbar machen                               \\ \hline
    H8 & Kleine Hinweise auf Probleme geben                              \\ \hline
    H9 & Lagerplätze als baubare Strukturen erkenntlicher machen                              \\ \hline
    H10 & Zentrale Informationen (Uhrzeit) kenntlicher darstellen                             \\ \hline
    H11 & Temperatur als mögliche Erweiterung der Komplexität                             \\ \hline
    H12 & Intuitive Converter als Erweiterung der Komplexität                             \\ \hline
    H13 & Prioritätensystem schafft mehr Kontrolle                             \\ \hline
    H14 & Prioritäten in Form von Grafiken erhöhen das Verständnis                             \\ \hline
    H15 & Das indirekte Anweisen der Kolonisten verstärkt das Spielgefühl positiv                            \\ \hline
    H16 & Die Steuerung sollte ersichtlicher sein   \\ \hline
    H17 & Zufällige Startbedingungen erhöhen die replayability                            \\ \hline
    H18 & Grafiken erhöhen die Übersicht im User-Interface                            \\ \hline
    H19 & Die Schwierigkeit muss sinnvoll zwischen herausfordernd und entspannt balanciert werden                            \\ \hline
    H20 & Es bedarf einem Tutorial und/oder Hilfestellungen auf Abruf                            \\ \hline
    H21 & Eine Zeitkontrolle ist eine sinnvolle Ergänzung                            \\ \hline
    \end{tabular}
    \end{table}