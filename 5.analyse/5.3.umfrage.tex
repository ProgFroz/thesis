\newpage
\subsection{Umfrage}
Die zuvor erarbeiteten Hypothesen sollten ursprünglich im Folgenden über die empirische Methode der Online-Umfrage überprüft werden \cite*[]{Evans2005TheVO}.  Die Online-Umfrage sollte auf \textit{Discord}-Servern durchgeführt werden. Discord ist ein soziales Medium, welches dazu dient Videos oder Videos auszutauschen, oder sich über Sprach- oder Textkanäle miteinander zu vernetzen und zu kommunizieren \cite*[]{discord:usage}. Discord hat 140 Millionen aktive Nutzer pro Monat, Tendenz stark steigend (Stand 2021) \cite*[]{discord:statistics}. Es gibt dabei mehrere \textit{Discord-Server}, welchen man mit einer Einladung beitreten kann, auf welchem sich mehrere Personen befinden, mit welchen man kommunizieren kann. Es gibt für RimWorld einen dedizierten Discord-Server, auf welchem sich (Stand 15.08.2022) 37.634 Mitglieder befinden. Bei der Erarbeitung ist jedoch ein entscheidender Punkt aufgefallen, welcher die Umfrage überflüssig macht: Discord-Server dieser Art werden im Normalfall tendenziell eher von Leuten betreten, welche bereits viel Kontakt mit dem Spiel haben oder hatten, und Informationen austauschen wollen. Da die Hypothesen zu einem entscheidenden Teil aus Annahmen bestehen, welche das Spielgefühl für Anfänger optimieren sollen, wäre der Raum, in welchem die Umfrage stattfinden würde, unangemessen. Ein erfahrener Spieler ist bereits an das User Interface gewöhnt, weiß, wonach und wohin er schauen muss und hat eigene Strategien für das Spiel entwickelt, wie das Interview mit Proband C gezeigt hat, was die Hypothesen großteils unbrauchbar macht.