\newpage
\section*{Glossar}
\paragraph*{AI Storyteller} Ein Algorithmus, der basierend auf derzeitigen Kontext die Entscheidungen über Events trifft
\paragraph*{Pangea-artig} Bezieht sich auf den Riesenkontinent Pangaea, eine große Landmasse mit Meer umgeben
\paragraph*{Skills} Fähigkeiten oder Aktionen
\paragraph*{Discord} Ein soziales Medium zur Kommunikation über Stimme und Text, wie auch Bilder oder Videos
\paragraph*{Sources} Im Kontext von Ressourcen eine Quelle für mehr Ressourcen
\paragraph*{Converter} Im Kontext von Ressourcen eine Umwandlung von einer Ressource in eine andere
\paragraph*{Drain} Im Kontext von Ressourcen eine Möglichkeit, Ressourcen aus dem Spiel zu entfernen
\paragraph*{Trader} Im Kontext von Ressourcen eine Möglichkeit, Ressourcen zu tauschen, wobei beide Parteien ein eigenständiges Inventar haben
\paragraph*{Inventar} Eine Möglichkeit zur Lagerung von Gegenständen, welche meist an einen einzelnen Charakter oder eine Gruppe von Charakteren gebunden ist.
\paragraph*{Enum} Kurz für \glqq enumeration\grqq\; und bezeichnet einen Datentypen zur Repräsentation von Zahlenwerten als Elemente einer begrenzten Menge.
\paragraph*{Asset Store} Online-Plattform von Unity für die Bereitstellung für Inhalte Dritter 
\paragraph*{Mockup} Ein Erstentwurf, meist in Form einer simplen Skizze, kann aber auch stark ausgearbeitet sein
\paragraph*{Unit Tests} Dienen dem Testen einzelner Funktionalitäten.
\paragraph*{Corner Case} Ein wenig wahrscheinlicher Fall 
\paragraph*{Bug} Ein Fehler im Quellcode 
\paragraph*{Game Engine} Eine Software zur Bereitstellung verschiedener Systeme, darunter Physik- oder Audiosysteme für die Entwicklung eines Spiels.
\paragraph*{Realtime Strategy Game} Echtzeit-Strategiespiel
\paragraph*{Turn Based} Rundenbasiert
\paragraph*{Isometrisch} Eine Art der Perspektive
\paragraph*{Domination} Im Kontext von Videospielen meist eine Siegbedingung, welche erfüllt wird durch das Zerstören von Gegnern
\paragraph*{Replayability} Wiederspielbarkeit, das Gefühl von erneuter Lust ein Spiel zu spielen
\paragraph*{SAT} SEGA Saturn, eine Konsole von SEGA
\paragraph*{Tangible} Greifbar, im Kontext von Ressourcen eine Ressource mit Koordinaten im Raum
\paragraph*{Intangible} Nicht greifbar, im Kontext von Ressourcen eine Ressource, welche sich nicht im Raum befindet
\paragraph*{Concrete} Konkret, im Kontext von Ressourcen eine Ressource, welche vom Spieler als solche wahrgenommen werden
\paragraph*{Abstract} Abstrakt, im Kontext von Ressourcen eine Ressource, welche vom Spieler nicht als solche wahrgenommen werden

\paragraph*{Grid} Ein Raster, welches aus bestimmten geometrischen Formen aufgebaut ist
