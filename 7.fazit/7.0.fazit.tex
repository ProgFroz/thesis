Es wurde binnen 8 Wochen ein Prototyp auf Grundlage von wissenschaftlicher Arbeit und kreativem Input konzipiert, implementiert und getestet. Es hat sich gezeigt, dass die ausgewählte Thematik der Bienenkolonie einen äußerst vielversprechender Kontext für kreative und mechanisch interessante Ideen darstellt, welche viel Potenzial für künftige Iterationen bieten. Einer der wohl wichtigsten Erkenntnisse ist, dass das Einbinden von Probanden mittels Interviews und User Testing in die Konzipierung des Spiels sehr große Vorteile mit sich bringt. Es zeigt sich damit, dass Methoden der nutzungsorientierten Gestaltung positive Einflüsse auf den Verlauf der Entwicklung eines Videospiels haben. Außerdem war es in Retrospektive die richtige Entscheidung, Probanden mit jeweils drei verschiedenen Erfahrungsgraden zu wählen, da die daraus gewonnen Ergebnisse sehr hilfreich für die Konzipierung des Spiels waren. Es existiert großes Potenzial für die Weiterentwicklung, wodurch dieses Projekt mit enormer Sicherheit weitergeführt wird. Es ist zum Stand der Arbeit noch nicht ersichtlich, wie viele Iterationen noch benötigt werden, um das Projekt abzuschließen. Allerdings waren es erfreulicherweise wenige Bugs, die im Laufe der Implementierung behoben werden mussten, was den Entwicklungsprozess deutlich vereinfacht hat. Es wurde ein Großteil der Hypothesen umgesetzt, wenn nicht praktisch dann trotzdem theoretisch im Konzept, was als Erfolg angesehen werden kann. Es besteht ein Gameplay Loop, ein Spielbeginn, ein Spielende und viele durchaus interessante Mechaniken, die das Spiel von anderen Titeln des Genres abheben, gleichzeitig aber auch auf gut funktionierende Elemente anderer Vertreter fundieren. Alles in allem ist die Arbeit, und damit auch das Projekt, ein voller Erfolg. Der Prototyp ist sowohl als bereits gebaute Version, wie auch als Unity Projekt auf dem Datenträger enthalten.