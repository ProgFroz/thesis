Videospiele integrieren sich mehr und mehr in den Alltag eines jeden, und ist auch seit meiner Kindheit ein Teil der Freizeit. Das erste Videospiel, das ich jemals spielen durfte, war \textit{Age of Empires}, kurze Zeit später auch \textit{Starcraft II} und \textit{Anno 1701}. Alle diese Spiele haben als zentrale Eigenschaft die Verwaltung von Ressourcen und erfordern damit ein taktisches Geschick und kreatives, effizientes Denken, was mich als Kind faszinierte. Der Erfolg des Spieles stand oder fiel mit der eigenen Effizienz, man musste erst Scheitern und aus den Fehlern lernen, um Fortschritte zu machen und bei dem nächsten Versuch die Hürde zu überwinden. Seitdem begeistert mich dieses Genre mit stetig neuen Ideen und Konzepten, welche neue Herausforderungen und Erfolge bieten. Die Palette der Spiele ist selbstverständlich deutlich zu groß, um in diesen wenigen Seiten erfasst werden zu können, aber diese Arbeit soll einen kleinen Einblick in das Genre an sich bieten, wie auch die Konzepte, die Entwicklung der Ideen und die letztendliche Implementierung.