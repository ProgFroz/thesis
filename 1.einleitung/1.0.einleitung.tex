Videospiele nehmen seit Jahren mehr und mehr Einzug in den Alltag der Gesellschaft. Immer mehr Kinder, Jugendliche, aber auch Erwachsene und Senioren finden Anklang an der Idee eines virtuellen Universums. Dies zeigt sich auch in den Verkaufszahlen, im Jahr 2021 betrug der Wert der globalen Gaming-Industrie \textit{\$214.2} Milliarden US-Dollar und wird auf einen Wert von \textit{\$321.1} Milliarden US-Dollar im Jahr 2026 geschätzt \cite*[]{gamesgrowth}. Videospiele sind demnach nicht nur ein Hobby oder eine spaßige Erfahrung, sondern auch aus einer kommerziellen Sicht interessant und daher stark relevant für die Industrie. Das Thema dieser Arbeit bezieht sich auf ein spezielles Genre der Videospiele, sogenannte \textit{Resource Management Games}. Das Ziel dieser Arbeit ist es, anhand Analysen und Methoden der nutzungsorientierten Gestaltung ein Spiel zu konzipieren und letztendlich einen Prototypen anzufertigen, der einige untersuchte und als gut beurteilte Elemente enthält. Dabei stehen die Fragen im Vordergrund, welche Elemente dieses Genres gut oder schlecht funktionieren, welche Mechaniken einsteigerfreundlich sind, und wie insgesamt ein Spiel von Anfang bis Ende mittels wissenschaftlicher Methodik erarbeitet werden kann. Es werden zu Beginn die theoretischen Konzepte des Genres, darunter die Fundamente einer Ökonomie, erläutert. Außerdem werden einige Beispiele mit den wichtigsten Eckdaten, beispielsweise Siegbedingungen, vorhandene Ressourcen, User-Interface Elemente und Mechaniken, aufgezeigt. In der anschließenden Analyse wird ein weiterer Titel des Genres näher untersucht. Es werden sowohl User Interface, als auch Mechaniken erörtert, wie auch mehrere Interviews mit gegebenen Fragen zu dem Spiel geführt. Mit diesen Interviews werden verschiedene Hypothesen interessanter Mechaniken und Elemente aufgestellt, welche zur Konzipierung des Prototypen verwendet werden. Des Weiteren wird die Thematik und der Kontext des Spiels etabliert, welcher sich an einer Bienenkolonie orientiert und dabei möglichst wissenschaftlich akkurat sein soll, wodurch auch einige biologische Phänomene in dem Konzept Einklang finden, darunter haploide und diploide Chromosomensätze. Es werden verschiedene Mechaniken erwogen und aufgezeigt, welche sich ebenfalls stark an die Vorgänge einer Bienenkolonie halten. Letztendlich wird der Prototyp in einer gewählten Umgebung implementiert, darunter die generierte Welt anhand einer erörterten Struktur und ein erarbeitetes User-Interface. Dieser Prototyp wird zuletzt in einem User Testing auf die Probe gestellt, wobei Konklusionen aufgestellt werden und gegebene Hypothesen unterstützt werden.

\subsection{Geschichte}
Bereits seit dem Jahr 1964 existieren Ressource Management Games. Das erste verzeichnete Spiel trägt dem Titel \glqq The Sumerian Game\grqq, wurde entwickelt von Mabel Addis und spielt in dem antiken sumerischen Stadtstaat Lagash. Die Aufgabe des Spielers war es, in drei verschiedenen Segmenten mit jeweils eigenen Runden, die Bevölkerung und die Ressourcen so anzuordnen, dass ein erfolgreiches Überleben der Einwohner gesichert ist, trotz zufälliger Katastrophen oder Events. Seitdem hat sich in der Branche des Game Developments und der Kategorie der Resource Management Games einiges getan, von SimCity über Anno, bis hin zu RimWorld oder Factorio. Es gibt verschiedenste Unterkategorien von Resource Management Games, welche alle eigene Siegbedingungen, Tücken und Eigenschaften besitzen, wie auch typische UI-Elemente und -Anordnungen. Die Branche ist erfolgreicher denn je und erfreut sich stetigem Wachstum, doch die Frage ist, in welche Richtung werden sich Resource Management Games entwickeln? Gibt es zurzeit Optimierungsbedarf an bestimmten Stellen? Was hält den Spieler dazu an, weiter das Spiel zu spielen, und wie kann man den Spaß optimieren? Was sind gängige Ressourcen, und was eher untypische? Gibt es eine (bisher) minimale und maximale Komplexität?
\subsection{Motivation}
Videospiele integrieren sich mehr und mehr in den Alltag eines jeden, und ist auch seit meiner Kindheit ein Teil der Freizeit. Das erste Videospiel, das ich jemals spielen durfte, war \textit{Age of Empires}, kurze Zeit später auch \textit{Starcraft II} und \textit{Anno 1701}. Alle diese Spiele haben als zentrale Eigenschaft die Verwaltung von Ressourcen und erfordern damit ein taktisches Geschick und kreatives, effizientes Denken, was mich als Kind faszinierte. Der Erfolg des Spieles stand oder fiel mit der eigenen Effizienz, man musste erst Scheitern und aus den Fehlern lernen, um Fortschritte zu machen und bei dem nächsten Versuch die Hürde zu überwinden. Seitdem begeistert mich dieses Genre mit stetig neuen Ideen und Konzepten, welche neue Herausforderungen und Erfolge bieten. Die Palette der Spiele ist selbstverständlich deutlich zu groß, um in diesen wenigen Seiten erfasst werden zu können, aber diese Arbeit soll einen kleinen Einblick in das Genre an sich bieten, wie auch die Konzepte, die Entwicklung der Ideen und die letztendliche Implementierung. Ursprünglich waren die Forschungsfragen eher auf die Komplexität des Genres fokussiert und sollte eine Spanne von minimaler bis maximaler Komplexität aufzeigen. Dieser Ansatz hat sich im Laufe der Arbeit jedoch als schwer messbar erwiesen und wenig praktisch für den Prototypen, womit dieser Fokus verschoben wurde.