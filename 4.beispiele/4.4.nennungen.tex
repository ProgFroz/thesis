\subsection{Weitere Beispiele}
Um den Überblick bestimmter Vertreter des Genres weiter auszubauen, werden im Folgenden weitere interessante Titel genannt, welche eigene Mechaniken innehaben und damit das Genre vorantreiben. Es ist 

\paragraph*{Factorio} Ein im August 2020 erschienenes Strategiespiel, welches von \textit{Wube Software LTD.} entwickelt und publiziert wurde \cite*[]{igdb:factorio} und einen großen Fokus auf Automatisierung und Effizienz legt. Der Spieler beginnt mit dem manuellen Abbau von Ressourcen, und beginnt allmählich primitive Prozesse zu automatisieren mittels Fliesbänder, Öfen und Kisten. Es gibt dabei etliche Produktionsketten mit verschiedenen Tücken und Tricks welche automatisiert werden können und dem Spieler Freiraum für kreatives Denken geben. Nebenbei ist der Spieler der Gefahr von außerirdischen Kreaturen ausgesetzt, welche ab und an angreifen und gegen welche er sich verteidigen muss.

\paragraph*{Frostpunk} Das von \textit{11 bit studios} entwickelte und publizierte Strategiespiel \cite*[]{igdb:frostpunk} fokussiert sich auf das Überleben im Eis. Überlebende Menschen in einer komplett gefrorenen Welt versuchen, eine Stadt zu gründen. Die Tücke dabei ist die Temperatur. Im Kern der Stadt befindet sich ein Generator, welcher stetig mit Kohle betrieben werden muss, da sonst die Menschen der Stadt erfrieren. Ressourcen sind sehr begrenzt und können außerdem über Spähtrupps von der Außenwelt gefunden und zurückgebracht werden, falls diese die Reise überleben. Der Spieler wird immer wieder mit Schneestürmen konfrontiert, welche stetig härter und schwieriger werden. Die Bevölkerung besitzt außerdem ein Level an Moral, welches aufrechterhalten werden muss. Fällt die Moral unter einen gewissen Punkt, endet das Spiel.

\paragraph*{Warcraft III} Veröffentlicht im Jahr 2002, entwickelt und publiziert von \textit{Blizzard Entertainment} \cite*[]{igdb:warcraft}. Das Spiel stellt einen Meilenstein des \textit{Real Time Strategy} Genres dar, wobei man zwischen vier verschiedenen Völkern wählen kann, welche allesamt eigene Einheiten, Eigenschaften und Mechaniken bieten. Außerdem gibt es verschiedene Heldeneinheiten, welche Gegenstände aufsammeln und tragen können und diese damit verstärken.

\paragraph*{Siedler von Catan} Das 1995 erschienene Brettspiel ist bis heute ein gespieltes Gesellschaftsspiel, welches von \textit{Klaus Teuber} entworfen wurde. Das Spielfeld ist in Hexagone aufgeteilt und rundenbasiert, wobei jedes Hexagon einer bestimmten Ressource zugeordnet ist. Die Spieler bauen zwischen den jeweiligen Hexagonen Siedlungen und im Verlauf des Spiels auch Städte, um auf diese Ressourcen Zugriff zu erhalten. 
